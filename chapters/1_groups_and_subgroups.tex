\chapter{Groups and Subgroups}

\section{Groups}

\begin{defn}
Let $ S $ be a set. A mapping
\begin{equation*}
    \begin{array}{rccc}
        \odot: & S\times S & \to & S \\
        & (x,y) & \mapsto & x\odot y
    \end{array}
\end{equation*}
is called a \defnem{law of composition} on $ S $.
\end{defn}

Note that $ S $ is necessarily closed under the operation defined by such a law. Examples include addition of natural numbers and multiplication of $ n\times n $ matrices. Subtraction of natural numbers, however, is not closed and therefore not a law of composition.

\begin{defn}
A law of composition $ \odot $ on $ S $ is called \defnem{associative} if for every $ x,y,z\in S $, we have $ (x\odot y)\odot z=x\odot(y\odot z) $. The law $ \odot $ is called \defnem{commutative} if for every $ x,y\in S $, we have $ x\odot y=y\odot x $.
\end{defn}

\begin{defn}
Let $ G $ be a set and $ \odot $ be a law of composition on $ G $. A pair $ (G,\odot) $ is called a \defnem{group} if
\begin{enumerate}
    \item $ \odot $ is associative;
    \item there exists a \defnem{neutral element} $ e\in G $ such that for every $ g\in G $, we have
    \begin{equation*}
        g\odot e=e\odot g=g;
    \end{equation*}
    and
    \item for every $ g\in G $, there exists an \defnem{inverse element} $ g^{-1}\in G $ such that
    \begin{equation*}
        g\odot g^{-1}=g^{-1}\odot g=e.
    \end{equation*}
\end{enumerate}
A group whose law is commutative is called \defnem{commutative} or \defnem{abelian}.
\end{defn}

We will typically refer to a group by its set and denote compositions of its elements using multiplicative notation $ ab $ if commutativity is not assumed, or additive notation $ a+b $ if commutativity is assumed; in the latter case, the inverse of $ a $ is denoted $ -a $.

\begin{prop}
The neutral element of a group is unique.
\end{prop}
\begin{proof}
Let $ G $ be a group, and let $ e_1,e_2\in G $ such that for every $ g\in G $, we have
\begin{equation*}
    e_1g=ge_1=g \quad\text{and}\quad e_2g=ge_2=g.
\end{equation*}
Then, $ e_1e_2=e_1 $ and $ e_1e_2=e_2 $, so $ e_1=e_2 $.
\end{proof}

\begin{prop}
Let $ G $ be a group. For every $ g\in G $, its inverse element $ g^{-1} $ is unique.
\end{prop}
\begin{proof}
Let $ g\in G $. Suppose $ h_1 $ and $ h_2 $ are both inverses of $ g $. Then,
\begin{equation*}
    gh_1=h_1g=e \quad\text{and}\quad gh_2=h_2g=e,
\end{equation*}
so
\begin{equation*}
    h_1=h_1e=h_1(gh_2)=(h_1g)h_2=eh_2=h_2.
\end{equation*}
Hence, the inverse of $ g $ is unique.
\end{proof}

\begin{prop}\label{prop:group_elems}
Let $ G $ be a group, and let $ g,h,i\in G $. Then,
\begin{enumerate}
    \item $ (g^{-1})^{-1}=g $;
    \item $ (gh)^{-1}=h^{-1}g^{-1} $;
    \item the equations $ gx=h $ and $ xg=h $ have unique solutions $ x\in G $; and
    \item if $ gi=hi $ or $ ig=ih $, then $ g=h $.
\end{enumerate}
\end{prop}

These can be proven with straightforward computations.

\section{Subgroups}

\begin{defn}
Let $ (G,\odot) $ be a group, and let $ H\subseteq G $. If $ (H,\odot\rvert_{H\times H}) $ is a group, it is called a \defnem{subgroup} of $ G $.
\end{defn}

\begin{thm}\label{thm:subgroup}
Let $ G $ be a group, and let $ H\subseteq G $, $ H\neq\varnothing $. Then, $ H $ is a subgroup of $ G $ if and only if for every $ h_1,h_2\in H $, we have $ h_1h_2^{-1}\in H $.
\end{thm}
\begin{proof}
\todo{Do this proof!}
\end{proof}

We will use the notation $ n\mathbb{Z}=\{n\cdot k\mid k\in\mathbb{Z}\} $ where $ \cdot $ is standard multiplication.

\begin{prop}
Let $ n\in\mathbb{Z} $. Then, $ (n\mathbb{Z},+) $ is a subgroup of $ (\mathbb{Z},+) $.
\end{prop}
\begin{proof}
We see $ 0\in n\mathbb{Z} $ for all $ n\in\mathbb{Z} $, so $ n\mathbb{Z}\neq\varnothing $.

Let $ a,b\in n\mathbb{Z} $. Then, $ a=kn $ and $ b=ln $ for some $ k,l\in\mathbb{Z} $, so we have
\begin{equation*}
    a+(-b)=a-b=kn-ln=(k-l)n=n(k-l)\in n\mathbb{Z}.
\end{equation*}
Hence, by Theorem \ref{thm:subgroup}, $ (n\mathbb{Z},+) $ is a subgroup of $ (\mathbb{Z},+) $.
\end{proof}

\begin{prop}
Every subgroup of $ (\mathbb{Z},+) $ is of the form $ (n\mathbb{Z},+) $ for some $ n\in\mathbb{Z} $.
\end{prop}
\begin{proof}
Let $ H $ be a subgroup of $ (\mathbb{Z},+) $. If $ H=\{0\} $, then $ H=0\mathbb{Z} $. Otherwise, let $ k\in H $, $ k\neq 0 $. Without loss of generality, take $ k $ to be positive. Now let $ S=H\cap\mathbb{Z}^+ $. Since $ k\in S $, we see $ S\neq\varnothing $, so $ S $ has a minimal element, say $ n $.

Since $ n\in H $, we see $ n\mathbb{Z}\subseteq H $. Additionally, rewriting $ k $ in terms of its Euclidean division by $ n $ as $ k=ln+r $ where $ l,r\in\mathbb{N}\cup\{0\} $, $ 0\leq r<n $, we see $ r=0 $ since $ n $ is minimal. Thus, $ k=ln\in n\mathbb{Z} $, so $ H\subseteq n\mathbb{Z} $. Hence, $ H=n\mathbb{Z} $.
\end{proof}

\begin{prop}\label{prop:generating_set}
Let $ G $ be a group, and let $ S\subseteq G $. Then, there exists a unique subgroup $ H $ of $ G $ such that
\begin{enumerate}
    \item $ S\subseteq H $ and \label{prop:generating_set.1}
    \item if $ H' $ is a subgroup of $ G $ and $ S\subseteq H' $, then $ H $ is a subgroup of $ H' $. \label{prop:generating_set.2}
\end{enumerate}
\end{prop}
\begin{proof}[Proof A]
Let $ X $ be the set of all subgroups of $ G $ that contain $ S $. Since $ G\in X $, we see $ X\neq\varnothing $. Now let $ H=\cup_{J\in X}J $. Then, $ S\subseteq H $. Finally, let $ x,y\in H $. Then, $ x,y\in J $ for all $ J\in X $, and since each $ J $ is a subgroup of $ G $, we have $ xy^{-1}\in J $ for all $ J\in X $. Thus,
\begin{equation*}
    xy^{-1}\in\bigcup_{J\in X}J=H,
\end{equation*}
so, by Theorem \ref{thm:subgroup}, $ H $ is a subgroup of $ G $.

Now suppose there exist two subgroups $ H_1,H_2 $ satisfying \ref{prop:generating_set.1} and \ref{prop:generating_set.2}. Then, $ S\subseteq H_1 $ and $ S\subseteq H_2 $. Since $ H_2 $ is a subgroup of $ G $ containing $ S $, by \ref{prop:generating_set.2} we have $ H_1\subseteq H_2 $; likewise, $ H_2\subseteq H_1 $, so $ H_1=H_2 $. Hence, $ H $ is unique.
\end{proof}
Alternatively, we can use a constructive proof:
\begin{proof}[Proof B]
Let $ H=\{g_1^{\pm 1}g_2^{\pm 1}\cdots g_k^{\pm 1}\mid g_1,g_2,\ldots,g_k\in S\} $. Then, $ S\subseteq H $. Further, let $ x,y\in H $. Then, $ x=g_1^{\pm 1}g_2^{\pm 1}\cdots g_n^{\pm 1} $ and $ y=h_1^{\pm 1}h_2^{\pm 1}\cdots h_m^{\pm 1} $ for some $ g_1,g_2,\ldots,g_n,h_1,h_2,\ldots,h_m\in S $, so
\begin{align*}
    xy^{-1} &= g_1^{\pm 1}g_2^{\pm 1}\cdots g_n^{\pm 1}(h_1^{\pm 1}h_2^{\pm 1}\cdots h_m^{\pm 1})^{-1} \\
    &= g_1^{\pm 1}g_2^{\pm 1}\cdots g_n^{\pm 1}(h_m^{\pm 1})^{-1}\cdots(h_2^{\pm 1})^{-1}(h_1^{\pm 1})^{-1} \\
    &= g_1^{\pm 1}g_2^{\pm 1}\cdots g_n^{\pm 1}h_m^{\mp 1}\cdots h_2^{\mp 1}h_1^{\mp 1}\in H.
\end{align*}
Thus, $ H $ is a subgroup of $ G $. Uniqueness can be shown in the same way as in Proof A.
\end{proof}

\begin{defn}
The subgroup $ H $ from Proposition \ref{prop:generating_set} is called the subgroup \defnem{generated by} $ S $, denoted $ \langle S\rangle $. This is, in other words, the smallest subgroup of $ G $ that contains $ S $. When $ \langle S\rangle=G $ for some group $ G $, we say $ S $ \defnem{generates} $ G $. When this $ S $ is finite, we say $ G $ is \defnem{finitely generated}.
\end{defn}

\begin{defn}
A group generated by one element, say $ x $, is called a \defnem{cyclic group}, denoted $ \langle x\rangle $.
\end{defn}

We will use the notation $ x^n $ to denote an element $ x $ of a group composed with itself $ n $ times.

\begin{prop}
Let $ G $ be a group, and let $ g\in G $. Then,
\begin{enumerate}
    \item $ \langle g\rangle=\langle\{g\}\rangle=\{g^m\mid m\in\mathbb{Z}\} $;
    \item $ \langle g\rangle $ is infinite if and only if there does not exist an $ m\in\mathbb{N} $ such that $ g^m=e $; and \label{prop:cyclic_groups.2}
    \item if $ \langle g\rangle $ is finite, then $ \lvert\langle g\rangle\rvert=\min\{m\in\mathbb{N}\mid g^m=e\} $.
\end{enumerate}
\end{prop}
\begin{proof}~
\begin{enumerate}

\item Since $ \langle g\rangle $ is a group, it must contain all compositions of $ g $ with itself, i.e. $ g^m $ for all $ m\in\mathbb{N} $, as well as its inverse $ g^{-1} $ and the inverses of those compositions, so at the minimum, $ \langle g\rangle $ contains $ \{g^m\mid m\in\mathbb{Z}\} $, which is a subgroup of $ G $. Hence, $ \langle g\rangle=\{g^m\mid m\in\mathbb{Z}\} $.

\item Suppose $ \langle g\rangle $ is finite. Equivalently, there exist some $ n,k\in\mathbb{Z} $, $ n\neq k $ such that $ g^n=g^k $; without loss of generality, take $ n>k $. We see
\begin{equation*}
    g^n=g^k \iff g^ng^{-k}=g^kg^{-k} \iff g^{n-k}=e,
\end{equation*}
i.e. there exists an $ m=n-k\in\mathbb{N} $ such that $ g^m=e $. Hence, $ \langle g\rangle $ is infinite if and only if such an $ m $ does not exist.

\item From the proof for \ref{prop:cyclic_groups.2}, it follows that if $ \langle g\rangle $ is finite, then the set $ \{m\in\mathbb{N}\mid g^m=e\} $ is nonempty and therefore has a least element, say $ n $. We see $ \{e,g,g^2,\ldots,g^{n-1}\}\subseteq\langle g\rangle $. Let $ g^k\in\langle g\rangle $ for some $ k\in\mathbb{Z} $. We can rewrite $ k $ in terms of its Euclidean division by $ n $ as $ k=nq+r $ for some $ q,r\in\mathbb{Z} $, $ 0\leq r<k$, giving us
\begin{equation*}
    g^k=g^{nq+r}=(g^n)^qg^r=e^qg^r=g^r\in\{e,g,g^2,\ldots,g^{n-1}\},
\end{equation*}
so $ \langle g\rangle\subseteq\{e,g,g^2,\ldots,g^{n-1}\} $. Hence, $ \langle g\rangle=\{e,g,g^2,\ldots,g^{n-1}\} $, so $ \lvert\langle g\rangle\rvert=n $.\qedhere
    
\end{enumerate}
\end{proof}

\begin{defn}
Let $ x $ be some element in a group. Then, $ \lvert\langle x\rangle\rvert $ is called the \defnem{order} of $ x $, denoted $ \ord(x) $.
\end{defn}

\section{Cosets}

\begin{defn}
Let $ G $ be a group and $ H $ be a subgroup of $ G $, and let $ g\in G $. Then, the set
\begin{equation*}
    gH=\{gh\mid h\in H\}
\end{equation*}
is called the \defnem{left coset} of $ H $ associated with $ g $, and the set
\begin{equation*}
    Hg=\{hg\mid h\in H\}
\end{equation*}
is called the \defnem{right coset} of $ H $ associated with $ g $.
\end{defn}

\begin{thm}\label{thm:coset_relations}
Let $ G $ be a group and $ H $ be a subgroup of $ G $, and let $ x,y\in G $. Then, the relations $ \sim_l $ and $ \sim_r $ on $ G $ such that
\begin{equation*}
    x\sim_l y\iff x^{-1}y\in H \quad\text{and}\quad x\sim_r y\iff xy^{-1}\in H
\end{equation*}
are equivalence relations.
\end{thm}
\begin{proof}
By Definition \ref{defn:equiv_relation}, we have three criteria for $ \sim_l $ to be an equivalence relation:
\begin{enumerate}
    \item We see $ x^{-1}x=e\in H $, so $ x\sim_l x $ (reflexive).
    \item Suppose $ x\sim_l y $. Then, $ x^{-1}y\in H $, so $ (x^{-1}y)^{-1}=y^{-1}x\in H $; therefore, $ y\sim_l x $ (symmetric).
    \item Let $ z\in G $. Suppose $ x\sim_l y $ and $ y\sim_l z $. Then, $ x^{-1}y,y^{-1}z\in H $, so
    \begin{equation*}
        (x^{-1}y)(y^{-1}z)=x^{-1}(yy^{-1})z=x^{-1}z\in H;
    \end{equation*}
    therefore, $ x\sim_l z $ (transitive).
\end{enumerate}
Thus, $ \sim_l $ is an equivalence relation. The same for $ \sim_r $ can be proven similarly.
\end{proof}

\begin{cor}[Alternative definition of the left and right cosets]
Let $ G $ be a group and $ H $ be a subgroup of $ G $, and take $ \sim_l $ and $ \sim_r $ as defined in Theorem \ref{thm:coset_relations}. Then, the left cosets of $ H $ in $ G $ are the equivalence classes of $ \sim_l $, and the right cosets are the equivalence classes of $ \sim_r $.
\end{cor}

\begin{cor}\label{cor:cosets_partition}
Let $ G $ be a group and $ H $ be a subgroup of $ G $. The left cosets of $ H $ in $ G $ form a partition of $ G $. The same applies for the right cosets.
\end{cor}

We will use the notation $ G/H $ to denote to denote the set of left cosets of $ H $ in $ G $ and $ H\backslash G $ to denote the set of right cosets.

\begin{prop}
Let $ G $ be a group and $ H $ be a subgroup of $ G $. Then, there exists a bijection between $ G/H $ and $ H\backslash G $. It follows that the number of left and right cosets is the same when finite.
\end{prop}
\begin{proof}
\todo{Do this proof!}
\end{proof}

\begin{defn}
Let $ G $ be a group and $ H $ be a subgroup of $ G $. The cardinality of $ G/H $ is called the \defnem{index} of $ H $ in $ G $, denoted $ [G:H] $.
\end{defn}

\begin{prop} \label{prop:cosets_cardinality}
Let $ G $ be a group and $ H $ be a subgroup of $ G $. Then, there exists a bijection between any two cosets of $ H $ in $ G $. It follows that if $ H $ is finite, then all the cosets are finite and have the same cardinality.
\end{prop}
\begin{proof}
Let $ g\in G $, and let
\begin{equation*}
    \begin{array}{rccc}
        f_g: & H & \to & gH \\
        & h & \mapsto & gh
    \end{array}.
\end{equation*}
By the definition of $ gH $, the mapping $ f_g $ is well-defined and surjective. Let $ h,h'\in H $ such that $ gh=gh' $. Then, by Proposition \ref{prop:group_elems}, we see $ h=h' $, so $ f_g $ is injective. Hence, $ f_g $ is a bijection, so $ \lvert H\rvert=\lvert gH\rvert $ when finite.
\end{proof}

\begin{thm}[Lagrange's theorem]
Let $ G $ be a finite group and $ H $ be a subgroup of $ G $. Then, the order of every subgroup of $ H $ divides the order of $ G $.
\end{thm}
\begin{proof}
By Corollary \ref{cor:cosets_partition}, we see $ G $ is the union of the left cosets, which are necessarily disjoint, so $ \lvert G\rvert $ is the sum of the cardinalities of the cosets. By Proposition \ref{prop:cosets_cardinality}, the cardinalities of the cosets are the same and equal to $ \lvert H\rvert $, so
\begin{equation*}
    \lvert G\rvert=[G:H]\lvert H\rvert. \qedhere
\end{equation*}
\end{proof}

\begin{cor}
Let $ G $ be a group and $ H,K $ be subgroups of $ G $ where $ K\subseteq H $. Then,
\begin{equation*}
    [G:K]=[G:H][H:K].
\end{equation*}
\end{cor}

\begin{cor}
Let $ G $ be a finite group, and let $ g\in G $. Then, $ \ord(g) $ divides $ \lvert G\rvert $. It follows that $ g^{\lvert G\rvert}=e $.
\end{cor}

\begin{cor}
Let $ G $ be a group of prime order. Then, $ G $ is cyclic; in other words, $ G=\langle g\rangle $ for all $ g\in G\setminus\{e\} $.
\end{cor}

\section{Normal subgroups}

\begin{defn}
Let $ G $ be a finite group and $ H $ be a subgroup of $ G $. If for every $ g\in G $, we have $ gH=Hg $, i.e. the left and right cosets are the same, then $ H $ is called a \defnem{normal subgroup} of $ G $.
\end{defn}

\begin{prop}
Let $ G $ be a finite group and $ H $ be a subgroup of $ G $. Then, $ H $ is a normal subgroup of $ G $ if and only if for every $ g\in G $ and $ h\in H $, we have $ ghg^{-1}\in H $.
\end{prop}
\begin{proof}~
\begin{enumerate}

\item[($ \Rightarrow $)] Suppose $ H $ is a normal subgroup of $ G $. Then, for all $ g\in G $, we have $ gH=Hg $, so for all $ h\in H $, we have $ gh\in Hg $. This means there exists some $ k\in H $ such that $ gh=kg $, so
\begin{equation*}
    ghg^{-1}=kgg^{-1}=k\in H.
\end{equation*}

\item[($ \Leftarrow $)] Let $ x\in gH $, and suppose for every $ g\in G $ and $ h\in H $, we have $ ghg^{-1}\in H $. Then, there exists some $ h\in H $ such that
\begin{equation*}
    x=gh=gh(g^{-1}g)=(ghg^{-1})g\in Hg,
\end{equation*}
so $ Hg\subseteq gH $. Similarly, it can be shown that $ gH\subseteq Hg $; hence, $ gH=Hg $. \qedhere

\end{enumerate}
\end{proof}

\begin{thm}
Let $ (G,\odot) $ be a group and $ H $ be a normal subgroup of $ G $. Then, $ G/H $ can be given a group structure with the composition law
\begin{equation*}
    \begin{array}{rccc}
        \oslash: & G/H\times G/H & \to & G/H \\
        & (xH,yH) & \mapsto & (x\odot y)H
    \end{array}.
\end{equation*}
\end{thm}
\begin{proof}
Since $ H $  is a normal subgroup, $ \oslash $ is well-defined. Associativity and inverses follow from $ \odot $. Since $ H=e_GH $, we have, for all $ gH\in G/H $,
\begin{equation*}
    H\oslash gH=e_GH\oslash gH=(e_G\odot g)H=gH,
\end{equation*}
and, similarly, $ gH\oslash H=gH $, so we have the neutral element $ H $. Hence, $ (G/H,\oslash) $ is a group.
\end{proof}

\sectionnumberless{Solved exercises}

Determine whether the following are groups, and show why or why not.
\begin{exer}
Consider $ (\{1,0,-1\},+) $ where $ + $ is standard addition.
\end{exer}
\begin{sltn}
Notice $ 1+1=2\notin\{1,0,-1\} $, so $ (\{1,0,-1\},+) $ is not a group.
\end{sltn}

\begin{exer}
Consider $ (\mathbb{R},\odot) $ where $ \odot $ is defined such that for $ x,y\in\mathbb{R} $, we have $ x\odot y=xy+(x^2-1)(y^2-1) $.
\end{exer}
\begin{sltn}
Notice
\begin{align*}
    2\odot(3\odot 4) &= 2\odot\left((3)(4)+(3^2-1)(4^2-1)\right)=2\odot 132 \\
    &= (2)(132)+(2^2-1)(132^2-1)=52\,533
\end{align*}
while
\begin{align*}
    (2\odot 3)\odot 4 &= \left((2)(3)+(2^2-1)(3^2-1)\right)\odot 4=30\odot 4 \\
    &= (30)(4)+(30^2-1)(4^2-1)=13\,605,
\end{align*}
so $ \odot $ is not associative. Hence, $ (\mathbb{R},\odot) $ is not a group.
\end{sltn}

\begin{exer}
Consider $ (\mathbb{R}^+,\odot) $ where $ \odot $ is defined such that for $ x,y\in\mathbb{R}^+ $, we have $ x\odot y=\sqrt{x^2+y^2} $.
\end{exer}
\begin{sltn}
Notice that for all $ x\in\mathbb{R}^+ $,
\begin{equation*}
    x\odot 0=\sqrt{x^2+0^2}=\sqrt{x^2}=x,
\end{equation*}
so $ 0 $ is the neutral element under $ \odot $; however, $ 0\notin\mathbb{R}^+ $, so $ (\mathbb{R}^+,\odot) $ is not a group.
\end{sltn}

\begin{exer}
Consider $ (\mathbb{R}\setminus\{-1\},\odot) $ where $ \odot $ is defined such that for $ x,y\in\mathbb{R}\setminus\{-1\} $, we have $ x\odot y=x+y+xy $.
\end{exer}
\begin{sltn}
Suppose there exists a pair $ (x,y) $ such that $ x\odot y=-1 $. Then,
\begin{align*}
    x+y+xy &= -1 \\
    y(1+x) &= -1-x \\
    y &= -\frac{1+x}{1+x} \\
    y &= -1
\end{align*}
so such a pair cannot be in $ (\mathbb{R}\setminus\{-1\})\times(\mathbb{R}\setminus\{-1\}) $; thus, $ \odot $ is a law of composition on $ \mathbb{R}\setminus\{-1\} $. We also see
\begin{align*}
    (x\odot y)\odot z &= (x+y+xy)\odot z=(x+y+xy)+z+(x+y+xy)z \\
    &= x+y+xy+z+xz+yz+xyz \\
    &= x+(y+z+yz)+x(y+z+yz)=x\odot(y+z+yz) \\
    &= x\odot(y\odot z)
\end{align*}
so $ \odot $ is associative. Finally, notice that for all $ x\in\mathbb{R}\setminus\{-1\} $, we have
\begin{equation*}
    x\odot 0=x+0+x(0)=x
\end{equation*}
(neutral element), and
\begin{align*}
    x\odot -\frac{x}{1+x} &= x-\frac{x}{1+x}+x\left(-\frac{x}{1+x}\right)=x-\frac{x}{1+x}-\frac{x^2}{1+x} \\
    &= \frac{x(1+x)-x}{1+x}-\frac{x^2}{1+x}=\frac{x^2}{1+x}-\frac{x^2}{1+x}=0
\end{align*}
(inverse). Hence, $ (\mathbb{R}\setminus\{-1\},\odot) $ is a group.
\end{sltn}

\begin{exer}
Consider $ (\mathcal{C},\cdot) $ where $ \mathcal{C}=\{z\in\mathbb{C}\mid\lvert c\rvert=1\} $ and $ \cdot $ is standard multiplication.
\end{exer}
\begin{sltn}
Since $ \mathcal{C} $ is the unit circle, we can uniquely represent each $ z\in\mathcal{C} $ in polar form as $ z=e^{i\theta} $ for some $ \theta\in(-\pi,\pi] $, and we know $ e^{i\theta}\in\mathcal{C} $ for all $ \theta\in\mathbb{R} $. Let $ e^{i\theta_1},e^{i\theta_2}\in\mathcal{C} $. Then,
\begin{equation*}
    e^{i\theta_1}\cdot e^{i\theta_2}=e^{i\theta_1+i\theta_2}=e^{i(\theta_1+\theta_2)}\in\mathcal{C}
\end{equation*}
so standard multiplication is a law of composition on $ \mathcal{C} $, and we know standard multiplication is associative. The neutral element under standard multiplication is $ 1=e^{i(0)}\in\mathcal{C} $. Finally, notice that for all $ e^{i\theta}\in\mathcal{C} $,
\begin{equation*}
    e^{i\theta}\cdot e^{i(-\theta)}=e^{i\theta-i\theta}=e^0=1
\end{equation*}
(inverse). Hence, $ (\mathcal{C},\cdot) $ is a group.
\end{sltn}

\begin{exer}
Consider $ (\mathrm{SL}_n(\mathbb{R}),\cdot) $ where $ \mathrm{SL}_n(\mathbb{R}) $ is the set of all $ n\times n $ matrices over $ \mathbb{R} $ with determinant 1 and $ \cdot $ is standard matrix multiplication.
\end{exer}
\begin{sltn}
Let $ A,B\in\mathrm{SL}_n(\mathbb{R}) $. Then,
\begin{equation*}
    \det(AB)=\det(A)\,\det(B)=(1)(1)=1
\end{equation*}
so $ AB\in\mathrm{SL}_n(\mathbb{R}) $. Thus, standard matrix multiplication is a law of composition on $ \mathrm{SL}_n(\mathbb{R}) $, and we know standard matrix multiplication is associative. The neutral element under standard matrix multiplication is $ I_n $ and $ \det(I_n)=1 $, so $ I_n\in\mathrm{SL}_n(\mathbb{R}) $. Finally, taking $ A^{-1} $ as the standard matrix inverse, we see
\begin{equation*}
    \det(A^{-1})=\frac{1}{\det(A)}=\frac{1}{1}=1
\end{equation*}
so $ A^{-1}\in\mathrm{SL}_n(\mathbb{R}) $. Hence, $ (\mathrm{SL}_n(\mathbb{R}),\cdot) $ is a group.
\end{sltn}

\begin{exer}
Consider $ (Q,\cdot) $ where $ Q=\{\pm I_2,\pm I,\pm J,\pm K\} $,
\begin{align*}
    I_2&=
    \begin{bmatrix}
        1 & 0 \\
        0 & 1
    \end{bmatrix}
    &
    I&=
    \begin{bmatrix}
        0 & 1 \\
        -1 & 0
    \end{bmatrix}
    &
    J&=
    \begin{bmatrix}
        0 & i \\
        i & 0
    \end{bmatrix}
    &
    K&=
    \begin{bmatrix}
        i & 0 \\
        0 & -i
    \end{bmatrix},
\end{align*}
and $ \cdot $ is standard matrix multiplication.
\end{exer}
\begin{sltn}
For $ I_2 $, $ I $, $ J $, and $ K $, we have the composition table
\begin{equation*}
    \begin{array}{c|cccc}
        \cdot & I_2  & I    & J    & K    \\
        \hline
        I_2   & I_2  & I    & J    & K    \\
        I     & I    & -I_2 & K    & -J   \\
        J     & J    & -K   & -I_2 & I    \\
        K     & K    & J    & -I   & -I_2
    \end{array}
\end{equation*}
and we know for any matrices $ A $ and $ B $,
\begin{equation*}
    (-A)B=A(-B)=-AB \quad\text{and}\quad (-A)(-B)=AB
\end{equation*}
so standard matrix multiplication is a law of composition on $ Q $. We also know standard matrix multiplication is associative. The neutral element under standard matrix multiplication of $ 2\times 2 $ matrices is $ I_2\in Q $. Finally, from the composition table, we have the inverses
\begin{align*}
    I_2^{-1}&=I_2 & I^{-1}&=-I & J^{-1}&=-J & K^{-1}&=-K \\
    \intertext{and from these we see}
    (-I_2)^{-1}&=-I_2 & (-I)^{-1}&=I & (-J)^{-1}&=J & (-K)^{-1}&=K.
\end{align*}
Hence, $ (Q,\cdot) $ is a group.
\end{sltn}

\begin{exer}
Consider $ (H,\cdot) $ where $ H $ is the set of upper triangular $ 3\times 3 $ matrices over $ \mathbb{R} $ with all $ 1 $s on the diagonal and $ \cdot $ is standard matrix multiplication.
\end{exer}
\begin{sltn}
Let $ a,b,c,x,y,z\in\mathbb{R} $. Then,
\begin{equation*}
    \begin{bmatrix}
        1 & x & y \\
        0 & 1 & z \\
        0 & 0 & 1
    \end{bmatrix}
    \begin{bmatrix}
        1 & a & b \\
        0 & 1 & c \\
        0 & 0 & 1
    \end{bmatrix}
    =
    \begin{bmatrix}
        1 & a+x & b+xc+y \\
        0 & 1 & c+z \\
        0 & 0 & 1
    \end{bmatrix}
    \in H
\end{equation*}
so standard matrix multiplication is a law of composition on $ H $, and we know standard matrix multiplication is associative. The neutral element under standard matrix multiplication of $ 3\times 3 $ matrices is $ I_3\in H $. Finally, computing the standard matrix inverse, we see
\begin{equation*}
    \begin{bmatrix}
        1 & x & y \\
        0 & 1 & z \\
        0 & 0 & 1
    \end{bmatrix}^{-1}
    =
    \begin{bmatrix}
        1 & -x & xz-y \\
        0 & 1 & -z \\
        0 & 0 & 1
    \end{bmatrix}
    \in H.
\end{equation*}
Hence, $ (H,\cdot) $ is a group.
\end{sltn}

For each of the following, determine whether $ H $ is a subgroup of $ G $, and show why or why not.

\begin{exer}
Let $ G=(\mathbb{R},+) $ and $ H=\{-1,0,1\} $.
\end{exer}
\begin{sltn}
Consider $ 1+1=2\notin H $. Hence, $ H $ is not a subgroup of $ G $.
\end{sltn}

\begin{exer}
Let $ G=(\mathbb{R},+) $ and $ H=\mathbb{R}\setminus\{0\} $.
\end{exer}
\begin{sltn}
The neutral element of $ G $ is $ 0\notin H $. Hence, $ H $ is not a subgroup of $ G $.
\end{sltn}

\begin{exer}
Let $ G=(\mathbb{C}\setminus\{0\},\cdot) $ and $ H=\mathbb{R}\setminus\{0\} $.
\end{exer}
\begin{sltn}
Let $ h_1,h_2\in H=\mathbb{R}\setminus\{0\} $. Then, since $ h_1,h_2\neq 0 $, we have
\begin{equation*}
    h_1h_2^{-1}=h_1\cdot\frac{1}{h_2}=\frac{h_1}{h_2}\in\mathbb{R}\setminus\{0\}=H.
\end{equation*}
Hence, $ H $ is a subgroup of $ G $.
\end{sltn}

\begin{exer}
Let $ G=(\mathbb{R}\setminus\{0\},\cdot) $ and $ H=\{-1,1\} $.
\end{exer}
\begin{sltn}
We see
\begin{align*}
    (-1)^{-1}&=\frac{1}{-1}=-1, & 1^{-1}&=\frac{1}{1}=1
\end{align*}
so
\begin{align*}
    -1\cdot(-1)^{-1}&=-1\cdot-1=1\in H, & -1\cdot 1^{-1}&=-1\cdot 1=-1\in H, \\
    1\cdot(-1)^{-1}&=1\cdot-1=-1\in H, & 1\cdot 1^{-1}&=1\cdot 1=1\in H.
\end{align*}
Hence, $ H $ is a subgroup of $ G $.
\end{sltn}

\begin{exer}
Let $ G=(\mathbb{C}\setminus\{0\},\cdot) $ and $ H=\left\{e^{i(2\pi k)/n}\mid k\in\{0,1,\ldots,n-1\}\right\} $ for some $ n\in\mathbb{N} $.
\end{exer}
\begin{sltn}
Let $ h_1,h_2\in H $. Then, $ h_1=e^{i(2\pi k)/n} $ and $ h_2=e^{i(2\pi l)/n} $ for some $ k,l\in\{0,1,\ldots,n-1\} $, so
\begin{equation*}
    h_2^{-1}=\left(e^{i(2\pi l)/n}\right)^{-1}=e^{-i(2\pi l)/n}
\end{equation*}
and we see
\begin{equation*}
    h_1h_2^{-1}=e^{i(2\pi k)/n}\cdot e^{-i(2\pi l)/n}=e^{i(2\pi(k-l))/n}.
\end{equation*}
Let $ m=(k-l)\bmod n $. Then,
\begin{equation*}
    h_1h_2^{-1}=e^{i(2\pi(k-l))/n}=e^{i(2\pi m)/n}\in H.
\end{equation*}
Hence, $ H $ is a subgroup of $ G $.
\end{sltn}

\begin{exer}
Let $ G=(\mathrm{GL}_n(\mathbb{R}),\cdot) $ where $ \mathrm{GL}_n(\mathbb{R}) $ is the set of all invertible $ n\times n $ matrices over $ \mathbb{R} $, and let $ H=(\mathrm{SL}_n(\mathbb{R}),\cdot) $.
\end{exer}
\begin{sltn}
Let $ A,B\in H=\mathrm{SL}_n(\mathbb{R}) $. Then,
\begin{equation*}
    \det(A)=\det(B)=1\neq0
\end{equation*}
so $ A^{-1} $ and $ B^{-1} $ exist and
\begin{equation*}
    \det(B^{-1})=\frac{1}{\det(B)}=\frac{1}{1}=1.
\end{equation*}
Therefore,
\begin{equation*}
    \det(AB^{-1})=\det(A)\det(B^{-1})=1\cdot 1=1
\end{equation*}
so $ AB^{-1}\in H $. Hence, $ H $ is a subgroup of $ G $.
\end{sltn}

\begin{exer}
Let $ G $ be a group, and let $ x\in G $ where $ x $ is of order $ k $. Prove that if $ m $ is an integer such that $ x^m=e_G $, then $ k\mid m $.
\end{exer}
\begin{sltn}
Since $ x $ is of order $ k $, we have by definition that $ k $ is the smallest positive integer such that $ x^k=e_G $. Suppose $ x^m=e_G $ for some $ m\in\mathbb{Z} $. We can rewrite $ m $ in terms of its Euclidean division by $ k $ as $ m=kn+r $ for some $ n,r\in\mathbb{Z} $ where $ 0\leq r<k $, giving us
\begin{equation*}
    x^m=x^{kn+r}=x^{kn}x^r=(x^k)^n x^r=e_G^n x^r=x^r.
\end{equation*}
so $ x^r=e_G $. Since $ r<k $, then $ r=0 $, so $ m=nk $. Hence, $ k\mid m $.
\end{sltn}