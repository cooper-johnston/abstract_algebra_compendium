\chapter{Rings and Fields}

\section{Rings}

\begin{defn}
Let $ R $ be a set, and let $ + $ and $ \cdot $ be two laws of composition on $ R $. The triple $ (R,+,\cdot) $ is called a \defnem{ring} if
\begin{enumerate}
    \item $ (R,+) $ is an abelian group;
    \item $ \cdot $ is associative; and
    \item $ \cdot $ is distributive over $ + $, i.e.\ for all $ x,y,z\in R $, we have
    \begin{equation*}
        (x+y)\cdot z=x\cdot z+y\cdot z \quad\text{and}\quad z\cdot(x+y)=z\cdot x+z\cdot y.
    \end{equation*}
\end{enumerate}
\end{defn}

Let $ (R,+,\cdot) $ be a ring, and let $ a,b\in R $. For the neutral element of $ R $ under $ + $, we will use the notation $ 0 $ or $ 0_R $; for the inverse of $ a $ under $ + $, we will use the notation $ -a $; and for the composition $ a\cdot b $, we will use the notation $ ab $. We will also assume the conventional order of operations, i.e.\ that $ \cdot $ comes before $ + $.

\begin{prop}
Let $ (R,+,\cdot) $ be a ring. Then,
\begin{enumerate}
    \item for all $ a\in R $, we have $ a(0)=0a=0 $; and
    \item for all $ a,b\in R $, we have
    \begin{equation*}
        a(-b)=(-a)b=-(ab) \quad\text{and}\quad (-a)(-b)=ab.
    \end{equation*}
\end{enumerate}
\end{prop}
\begin{proof}~
\begin{enumerate}
    \item We can rewrite $ 0 $ as $ 0+0 $ and use the distributive property:
    \begin{align*}
        a(0) &= a(0+0) & 0a &= (0+0)a \\
        a(0) &= a(0)+a(0) & 0a &= 0a+0a \\
        a(0)-(a(0)) &= a(0)+a(0)-(a(0)) & 0a-(0a) &= 0a+0a-(0a) \\
        0 &= a(0) & 0 &= 0a.
    \end{align*}

    \item Note that for any $ x,y\in R $, we have $ x=y $ if and only if $ x-y=0 $. Thus, since
    \begin{equation*}
        a(-b)+(ab)=a(-b+b)=a(0)=0,
    \end{equation*}
    we have $ a(-b)=-(ab) $. Similarly, we can show $ (-a)b=(-ab) $. By substitution, we then see
    \begin{equation*}
        (-a)(-b)-(ab)=(-a)(-b)+a(-b)=(-a+a)(-b)=0(-b)=0,
    \end{equation*}
    so $ (-a)(-b)=ab $.\qedhere
\end{enumerate}
\end{proof}

\begin{defn}\label{defn:ring_types}
A ring $ (R,+,\cdot) $ is called
\begin{enumerate}
    \item \defnem{commutative} if $ \cdot $ is commutative;
    \item a \defnem{ring with identity} if there exists some $ u\in R $ such that for every $ a\in R $, we have $ au=ua=a $; or
    \item an \defnem{integral domain} if it is a commutative ring with identity and for all $ a,b\in R $, if $ ab=0 $, then $ a=0 $ or $ b=0 $.
\end{enumerate}
\end{defn}

As with groups, we will also typically denote a ring $ (R,+,\cdot) $ simply by its set $ R $. We will also denote the element $ u\in R $ from Definition \ref{defn:ring_types} by $ 1 $ or $ 1_R $.

\begin{prop}
Let $ R $ be a ring with identity. Then,
\begin{enumerate}
    \item the element $ 1\in R $ is unique; and
    \item if there exist $ b,c\in R $ such that $ ab=ca=1 $ for some $ a\in R $, then $ b=c $.
\end{enumerate}
\end{prop}
\begin{proof}~
\begin{enumerate}
    \item Suppose there exist $ u,v\in R $ such that for every $ a\in R $, we have $ au=ua=a $ and $ av=va=a $. Then, in particular, $ u=uv=v $.

    \item By the associative property, we see
    \begin{equation*}
        b=1b=(ca)b=c(ab)=c(1)=c.\qedhere
    \end{equation*}
\end{enumerate}
\end{proof}

\begin{defn}
Let $ R $ be a commutative ring with identity. An element $ a\in R\setminus\{0\} $ is called a \defnem{zero divisor} if there exists some $ b\in R\setminus\{0\} $ such that $ ab=0 $.
\end{defn}

\begin{prop}
Let $ R $ be a commutative ring with identity. Then, the following are equivalent:
\begin{enumerate}
    \item $ R $ has no zero divisors;
    \item $ R $ is an integral domain;
    \item for every $ a,b,c\in R $ where $ a\neq 0 $, if $ ab=ac $, then $ b=c $.
\end{enumerate}
\end{prop}
\begin{proof}
Clearly, $ R $ is an integral domain if and only if $ R $ has no zero divisors. Now, let $ a\in R\setminus\{0\} $ and suppose for all $ b,c\in R $, we have
\begin{align*}
    ab &= ac \\
    ab-ac &= 0 \\
    a(b-c) &= 0.
\end{align*}
Since $ a\neq 0 $, we see by definition $ R $ is an integral domain if and only if this implies $ b-c=0 $ or, equivalently, $ b=c $.
\end{proof}

\begin{defn}
Let $ R $ be a ring with identity. An element $ a\in R $ is called a \defnem{unit} if there exists some $ a^{-1}\in R $ such that $ aa^{-1}=a^{-1}a=1 $. The set of units of $ R $ is denoted $ R^* $.
\end{defn}

\begin{prop}
Let $ R $ be a ring with identity. Then, $ (R^*,\cdot) $ is a group.
\end{prop}
\begin{proof}
We have three criteria for $ (R^*,\cdot) $ to be a group:
\begin{enumerate}
    \item Let $ a,b\in R^* $. Then, there exist some $ a^{-1},b^{-1}\in R $ such that
    \begin{equation*}
        aa^{-1}=a^{-1}a=1 \quad\text{and}\quad bb^{-1}=b^{-1}b=1.
    \end{equation*}
    Since associativity follows from the ring, we have
    \begin{align*}
        (ab)(b^{-1}a^{-1}) &= a(bb^{-1})a^{-1}=a(1)a^{-1}=aa^{-1}=1, \\
        (b^{-1}a^{-1})(ab) &= b^{-1}(a^{-1}a)b=b^{-1}(1)b=b^{-1}b=1.
    \end{align*}
    Thus, $ \cdot $ is an associative law of composition on $ R^* $.

    \item For every $ a\in R^* $, we have $ 1 a=a(1)=a $, so $ 1 $ is the neutral element.
    
    \item By construction, $ a^{-1} $ is then the inverse of $ a $.\qedhere
\end{enumerate}
\end{proof}

\begin{defn}
A ring $ R $ is called a \defnem{field} if it is a commutative ring with identity and all its nonzero elements are units, i.e.\ $ R\setminus\{0\}=R^* $.
\end{defn}

\begin{prop}
Let $ R $ be a ring with identity. Every unit of $ R $ is not a zero divisor.
\end{prop}
\begin{proof}
Let $ a\in R^* $. Then, there exists some $ a^{-1}\in R $ such that $ aa^{-1}=a^{-1}a=1 $. Suppose $ a $ is a zero divisor. Then, there exists some $ b\in R\setminus\{0\} $ such that $ ab=ba=0 $, so
\begin{equation*}
    (aa^{-1})b=(a^{-1}a)b=a^{-1}(ab)=a^{-1}(0)=0 \quad\text{and}\quad (aa^{-1})b=1b=b,
\end{equation*}
which implies $ b=0 $, a contradiction. Hence, $ a $ cannot be a zero divisor.
\end{proof}

\begin{cor}
Any field is an integral domain.
\end{cor}

\begin{thm}
Any finite integral domain is a field.
\end{thm}
\begin{proof}
Let $ R $ be a finite integral domain, and let $ a\in R\setminus\{0\} $. Consider the mapping
\begin{equation*}
    \begin{array}{rccc}
        f: & R & \to & R \\
        & x & \mapsto & ax
    \end{array}.
\end{equation*}
Let $ x,x'\in R $ such that $ ax=ax' $. Since $ R $ is an integral domain, left cancellation implies $ x=x' $, so $ f $ is injective. Further, since $ f $ is an injective map between finite sets of the same cardinality, $ f $ is also surjective, so there exists some $ b\in R $ such that $ f(b)=ab=1\in R $, and since an integral domain is necessarily commutative, we also have $ ba=1 $. Hence, $ a $ is a unit, so $ R\setminus\{0\}=R^* $.
\end{proof}

\begin{defn}
Let $ (R,+,\cdot) $ be a ring, and let $ S\subseteq R $. If $ (S,+,\cdot) $ is a ring, it is called a \defnem{subring} of $ R $.
\end{defn}

\begin{thm}
Let $ R $ be a ring, and let $ S\subseteq R $, $ S\neq\varnothing $. Then, $ S $ is a subring of $ R $ if and only if for every $ a,b\in S $, we have $ a-b\in S $ and $ ab\in S $.
\end{thm}
\begin{proof}
For $ S $ to be a ring, $ (S,+) $ must be an abelian group. Since $ S\subseteq R $, this is the case if and only if $ (S,+) $ is a subgroup of $ (R,+) $ which, by Theorem \ref{thm:subgroup}, is true if and only if for all $ a,b\in S $, we have $ a-b\in S $.

Associativity and distributivity of $ \cdot $ follow from the parent ring $ R $. Hence, all that remains is that $ S $ is closed under $ \cdot $, i.e.\ for all $ a,b\in S $, we have $ ab\in S $.
\end{proof}

\begin{defn}
Let $ R $ be a ring with identity, and let
\begin{equation*}
    K=\{n\in\mathbb{N}\mid\underbrace{1_R+\cdots+1_R}_{n\text{ times}}=0\}.
\end{equation*}
The number
\begin{equation*}
    \Char(R)=
    \left\{\begin{array}{ll}
        0, & K=\varnothing \\
        \min(K), & \text{otherwise}
    \end{array}\right.
\end{equation*}
is called the \defnem{characteristic} of $ R $.
\end{defn}

\begin{prop}
The characteristic of an integral domain is either $ 0 $ or prime.
\end{prop}
\begin{proof}
Let $ R $ be an integral domain, and let $ n=\Char(R) $. If $ n=0 $, we are finished; for the other case, since $ n $ cannot be 1, take $ n>1 $. Suppose $ n $ is not prime. Then, there exist $ p,q\in\mathbb{Z}^+ $, $ p,q<n $ such that $ n=pq $, so
\begin{align*}
    0_R &= \underbrace{1_R+\cdots+1_R}_{n\text{ times}}=\underbrace{1_R+\cdots+1_R}_{pq\text{ times}} \\
    &= \underbrace{(\underbrace{1_R+\cdots+1_R}_{p\text{ times}})+\cdots+(\underbrace{1_R+\cdots+1_R}_{p\text{ times}})}_{q\text{ times}} \\
    &= \underbrace{(\underbrace{1_R+\cdots+1_R}_{p\text{ times}})1_R+\cdots+(\underbrace{1_R+\cdots+1_R}_{p\text{ times}})1_R}_{q\text{ times}} \\
    &= (\underbrace{1_R+\cdots+1_R}_{p\text{ times}})(\underbrace{1_R+\cdots+1_R}_{q\text{ times}}).
\end{align*}
Since $ R $ is an integral domain, this implies
\begin{equation*}
    \underbrace{1_R+\cdots+1_R}_{p\text{ times}}=0_R \quad\text{or}\quad \underbrace{1_R+\cdots+1_R}_{q\text{ times}}=0_R,
\end{equation*}
which is a contradiction.
\end{proof}

\subsection*{Homomorphisms of rings}

\begin{defn}
Let $ (R,+,\cdot) $ and $ (S,\oplus,\odot) $ be two rings. A mapping $ \phi:R\to S $ is called a \defnem{homomorphism of rings} if for all $ x,y\in R $, we have
\begin{equation*}
    \phi(x+y)=\phi(x)\oplus\phi(y) \quad\text{and}\quad \phi(x\cdot y)=\phi(x)\odot\phi(y).
\end{equation*}
A homomorphism of rings that is a bijection is called an \defnem{isomorphism}.
\end{defn}

\begin{prop}\label{prop:underlying_morphism}
Let $ (R,+,\cdot) $ and $ (S,\oplus,\odot) $ be two rings. If there exists a homomorphism of rings $ \phi:R\to S $, then there exists a group homomorphism $ \psi:(R,+)\to(S,\oplus) $.
\end{prop}
\begin{proof}
\todo{Do this proof!}
\end{proof}

\begin{prop}
Let $ \phi:R\to S $ be a homomorphism of rings. Then, $ \phi $ is an isomorphism if and only if there exists a unique isomorphism $ \rho:S\to R $ such that $ \rho\circ\phi=\id_R $ and $ \phi\circ\rho=\id_S $.
\end{prop}
\begin{proof}
\todo{Do this proof!}
\end{proof}

\begin{defn}
Let $ \phi $ be a homomorphism of rings. The image and kernel of the underlying group homomorphism $ \psi $ from Proposition \ref{prop:underlying_morphism} are called the \defnem{image} and \defnem{kernel} of $ \phi $.
\end{defn}

\begin{prop}
Let $ \phi:R\to S $ be a homomorphism of rings. Then,
\begin{enumerate}
    \item $ \im(\phi) $ is a subring of $ S $;
    \item $ \ker(\phi) $ is a subring of R;
    \item $ \phi $ is injective if and only if $ \ker(\phi)=\{0_R\} $;
    \item $ \phi $ is surjective if and only if $ \im(\phi)=S $; and
    \item for every $ x\in R $ and $ y\in\ker(\phi) $, we have $ xy\in\ker(\phi) $.
\end{enumerate}
\end{prop}
\begin{proof}
\todo{Do this proof!}
\end{proof}

\section{Ideals}

\begin{defn}
Let $ R $ be a ring. A non-empty $ I\subseteq R $ is called an \defnem{ideal} of $ R $ if
\begin{enumerate}
    \item $ (I,+) $ is a subgroup of $ (R,+) $ and
    \item for all $ x\in R $ and $ i\in I $, we have $ xi\in I $ and $ ix\in I $.
\end{enumerate}
\end{defn}

\begin{defn}
Let $ R $ be a commutative ring with identity. An ideal $ I $ of $ R $ is called
\begin{enumerate}
    \item \defnem{prime} if for every $ x,y\in R $, if $ xy\in I $, then $ x\in I $ or $ y\in I $; or
    \item \defnem{maximal} if $ I\neq R $ and if there exists an ideal $ J $ such that $ I\subseteq J $, then $ I=J $ or $ J=R $.
\end{enumerate}
\end{defn}

\section{Arithmetic in integral domains}

\section{Polynomials}

\sectionnumberless{Solved exercises}