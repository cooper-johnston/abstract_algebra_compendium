\chapter{Rings and Fields}

\section{Rings}

\begin{defn}
Let $ R $ be a set, and let $ + $ and $ \cdot $ be two laws of composition on $ R $. The triple $ (R,+,\cdot) $ is called a \defnem{ring} if
\begin{enumerate}
    \item $ (R,+) $ is an abelian group;
    \item $ \cdot $ is associative; and
    \item $ \cdot $ is distributive over $ + $, i.e.\ for all $ x,y,z\in R $, we have
    \begin{equation*}
        (x+y)\cdot z=x\cdot z+y\cdot z \quad\text{and}\quad z\cdot(x+y)=z\cdot x+z\cdot y.
    \end{equation*}
\end{enumerate}
\end{defn}

For the group $ (R,+) $, we will use the notation $ 0 $ or $ 0_R $ for the neutral element and $ -a $ for the inverse of some $ a\in R $. On the ring, we will assume the conventional order of operations when writing expressions, i.e.\ that $ \cdot $ comes before $ + $.

\begin{prop}
Let $ (R,+,\cdot) $ be a ring. Then,
\begin{enumerate}
    \item for all $ a\in R $, we have $ a\cdot 0=0\cdot a=0 $; and
    \item for all $ a,b\in R $, we have
    \begin{equation*}
        a\cdot(-b)=(-a)\cdot b=-(a\cdot b) \quad\text{and}\quad (-a)\cdot(-b)=a\cdot b.
    \end{equation*}
\end{enumerate}
\end{prop}
\begin{proof}~
\begin{enumerate}
    \item By the distributive property, we have
    \begin{align*}
        a\cdot 0 &= a\cdot(0+0) & 0\cdot a &= (0+0)\cdot a \\
        a\cdot 0 &= a\cdot 0+a\cdot 0 & 0\cdot a &= 0\cdot a+0\cdot a \\
        a\cdot 0-(a\cdot 0) &= a\cdot 0+a\cdot 0-(a\cdot 0) & 0\cdot a-(0\cdot a) &= 0\cdot a+0\cdot a-(0\cdot a) \\
        0 &= a\cdot 0 & 0 &= 0\cdot a.
    \end{align*}

    \item Using the distributive property again, we have
    \begin{equation*}
        a\cdot(-b)+(a\cdot b)=a\cdot(-b+b)=a\cdot 0=0,
    \end{equation*}
    so $ a\cdot(-b)=-(a\cdot b) $. Similarly, $ (-a)\cdot b=(-a\cdot b) $. By substitution, we then see
    \begin{align*}
        (-a)\cdot(-b)-(a\cdot b) &= (-a)\cdot(-b)+a\cdot(-b)=(-a+a)\cdot(-b) \\
        &= 0\cdot(-b)=0,
    \end{align*}
    so $ (-a)\cdot(-b)=a\cdot b $.\qedhere
\end{enumerate}
\end{proof}

\begin{defn}\label{defn:ring_types}
A ring $ (R,+,\cdot) $ is called
\begin{enumerate}
    \item \defnem{commutative} if $ \cdot $ is commutative;
    \item a \defnem{ring with identity} if there exists a $ u\in R $ such that for every $ a\in R $, we have $ a\cdot u=u\cdot a=a $; or
    \item an \defnem{integral domain} if it is a commutative ring with identity and for all $ a,b\in R $, if $ a\cdot b=0 $, then $ a=0 $ or $ b=0 $.
\end{enumerate}
\end{defn}

As with groups, we will also typically denote a ring $ (R,+,\cdot) $ simply by its set $ R $. We will also denote the element $ u\in R $ from Definition \ref{defn:ring_types} by $ 1 $ or $ 1_R $.

\begin{prop}
Let $ R $ be a ring with identity. Then,
\begin{enumerate}
    \item the element $ 1_R $ is unique; and
    \item if there exist $ b,c\in R $ such that $ a\cdot b=c\cdot a=1_R $ for some $ a\in R $, then $ b=c $.
\end{enumerate}
\end{prop}
\begin{proof}~
\begin{enumerate}
    \item Suppose there exist $ u,v\in R $ such that for every $ a\in R $, we have $ a\cdot u=u\cdot a=a $ and $ a\cdot v=v\cdot a=a $. Then, in particular, $ u=u\cdot v=v $.

    \item By the associative property, we see
    \begin{equation*}
        b=1_R\cdot b=(c\cdot a)\cdot b=c\cdot (a\cdot b)=c\cdot 1_R=c.\qedhere
    \end{equation*}
\end{enumerate}
\end{proof}

\begin{defn}
Let $ R $ be a commutative ring with identity. An element $ a\in R\setminus\{0\} $ is called a \defnem{zero divisor} if there exists some $ b\in R\setminus\{0\} $ such that $ a\cdot b=0 $.
\end{defn}

\begin{prop}
Let $ R $ be a commutative ring with identity. Then, the following are equivalent:
\begin{enumerate}
    \item $ R $ has no zero divisors;
    \item $ R $ is an integral domain;
    \item for every $ a,b,c\in R $ where $ a\neq 0 $, if $ a\cdot b=a\cdot c $, then $ b=c $.
\end{enumerate}
\end{prop}
\begin{proof}
% We can use symbolic notation to show (1) and (3) are equivalent:
% \begin{align*}
%     R\text{ is an integral domain} &\iff (\forall a,x\in R,\, a\cdot x=0\implies a=0\vee x=0) \\
%     &\iff (\forall a,x\in R,\, a\cdot x=0\wedge a\neq 0\implies x=0) \\
%     &\iff (\forall a,b,c\in R,\, a\cdot(b-c)=0\wedge a\neq 0\implies b-c=0) \\
%     &\iff (\forall a,b,c\in R,\, a\cdot b=a\cdot c\wedge a\neq 0\implies b=c). \qedhere
% \end{align*}

Clearly, $ R $ is an integral domain if and only if $ R $ has no zero divisors. Now, let $ a,b,c\in R $ where $ a\neq 0 $ and suppose
\begin{align*}
    a\cdot b &= a\cdot c \\
    a\cdot b-a\cdot c &= 0 \\
    a\cdot(b-c) &= 0.
\end{align*}
Since $ a\neq 0 $, we see by definition $ R $ is an integral domain if and only if this implies $ b-c=0 $ or, equivalently, $ b=c $.
\end{proof}

\begin{defn}
Let $ R $ be a ring with identity. An element $ a\in R $ is called a \defnem{unit} if there exists some $ b\in R $ such that $ a\cdot b=b\cdot a=1 $. The set of units of $ R $ is denoted $ R^* $.
\end{defn}

\begin{prop}
Let $ R $ be a ring with identity. Then, $ (R^*,\cdot) $ is a group.
\end{prop}
\begin{proof}
We have three criteria for $ (R^*,\cdot) $ to be a group:
\begin{enumerate}
    \item Let $ a,x\in R^* $. Then, there exist some $ b,y\in R $ such that
    \begin{equation*}
        a\cdot b=b\cdot a=1 \quad\text{and}\quad x\cdot y=y\cdot x=1,
    \end{equation*}
    so
    \begin{align*}
        (a\cdot x)\cdot(y\cdot b) &= a\cdot(x\cdot y)\cdot b=a\cdot 1\cdot b=a\cdot b=1, \\
            (y\cdot b)\cdot(a\cdot x) &= y\cdot(b\cdot a)\cdot x=y\cdot 1\cdot x=y\cdot x=1.
    \end{align*}
    Thus, $ \cdot $ is a law of composition on $ R^* $, and we know $ \cdot $ is associative.

    \item For every $ a\in R^* $, we have $ 1\cdot a=a\cdot 1=a $, so $ 1 $ is the neutral element.
    
    \item By construction, $ b $ is then the inverse of $ a $.\qedhere
\end{enumerate}
\end{proof}

\begin{prop}
Let $ R $ be a ring with identity. Every unit of $ R $ is not a zero divisor.
\end{prop}
\begin{proof}
\todo{Do this proof!}
\end{proof}

\begin{defn}
A ring $ R $ is called a \defnem{field} if it is a commutative ring with identity and all its nonzero elements are units, i.e.\ $ R\setminus\{0\}=R^* $.
\end{defn}

\begin{thm}
Any finite integral domain is a field.
\end{thm}
\begin{proof}
Let $ R $ be a finite integral domain, and let $ a\in R\setminus\{0\} $. Consider the mapping
\begin{equation*}
    \begin{array}{rccc}
        f: & R & \to & R \\
        & x & \mapsto & a\cdot x
    \end{array}.
\end{equation*}
Let $ x,x'\in R $ such that $ a\cdot x=a\cdot x' $. Since $ R $ is an integral domain, left cancellation implies $ x=x' $, so $ f $ is injective. Further, since $ f $ is an injective map between finite sets of the same cardinality, $ f $ is also surjective, so there exists a $ b\in R $ such that $ a\cdot b=1\in R $, and since an integral domain is necessarily commutative, we also have $ b\cdot a=1 $. Hence, $ a $ is a unit, so $ R\setminus\{0\}=R^* $.
\end{proof}

\begin{defn}
Let $ (R,+,\cdot) $ be a ring, and let $ S\subseteq R $. The triple $ (S,+,\cdot) $ is called a \defnem{subring} of $ R $ if it itself is a ring.
\end{defn}

\begin{thm}
Let $ R $ be a ring, and let $ S\subseteq R $, $ S\neq\varnothing $. Then, $ S $ is a subring of $ R $ if and only if for every $ a,b\in S $, we have $ a-b\in S $ and $ a\cdot b\in S $.
\end{thm}
\begin{proof}
\todo{Do this proof!}
\end{proof}

\begin{defn}
Let $ R $ be a ring with identity, and let
\begin{equation*}
    K=\{n\in\mathbb{N}\mid\underbrace{1+\cdots+1}_{n\text{ times}}=0\}.
\end{equation*}
The number
\begin{equation*}
    \Char(R)=
    \left\{\begin{array}{ll}
        0, & K=\varnothing \\
        \min(K), & \text{otherwise}
    \end{array}\right.
\end{equation*}
is called the \defnem{characteristic} of $ R $.
\end{defn}

\begin{prop}
The characteristic of an integral domain is either $ 0 $ or prime.
\end{prop}
\begin{proof}
\todo{Do this proof!}
\end{proof}

\subsection*{Homomorphisms of rings}

\begin{defn}
Let $ (R,+,\cdot) $ and $ (S,\oplus,\odot) $ be two rings. A mapping $ \phi:R\to S $ is called a \defnem{homomorphism of rings} if for all $ x,y\in R $, we have
\begin{equation*}
    \phi(x+y)=\phi(x)\oplus\phi(y) \quad\text{and}\quad \phi(x\cdot y)=\phi(x)\odot\phi(y).
\end{equation*}
A homomorphism of rings that is a bijection is called an \defnem{isomorphism}.
\end{defn}

\begin{prop}\label{prop:underlying_morphism}
Let $ (R,+,\cdot) $ and $ (S,\oplus,\odot) $ be two rings. If there exists a homomorphism of rings $ \phi:R\to S $, then there exists a group homomorphism $ \psi:(R,+)\to(S,\oplus) $.
\end{prop}
\begin{proof}
\todo{Do this proof!}
\end{proof}

\begin{prop}
Let $ \phi:R\to S $ be a homomorphism of rings. Then, $ \phi $ is an isomorphism if and only if there exists a unique isomorphism $ \rho:S\to R $ such that $ \rho\circ\phi=\id_R $ and $ \phi\circ\rho=\id_S $.
\end{prop}
\begin{proof}
\todo{Do this proof!}
\end{proof}

\begin{defn}
Let $ \phi $ be a homomorphism of rings. The image and kernel of the underlying group homomorphism $ \psi $ from Proposition \ref{prop:underlying_morphism} are called the \defnem{image} and \defnem{kernel} of $ \phi $.
\end{defn}

\begin{prop}
Let $ \phi:R\to S $ be a homomorphism of rings. Then,
\begin{enumerate}
    \item $ \im(\phi) $ is a subring of $ S $;
    \item $ \ker(\phi) $ is a subring of R;
    \item $ \phi $ is injective if and only if $ \ker(\phi)=\{0_R\} $;
    \item $ \phi $ is surjective if and only if $ \im(\phi)=S $; and
    \item for every $ x\in R $ and $ y\in\ker(\phi) $, we have $ x\cdot y\in\ker(\phi) $.
\end{enumerate}
\end{prop}
\begin{proof}
\todo{Do this proof!}
\end{proof}

\section{Ideals}

\section{Arithmetic in integral domains}

\section{Polynomials}

\sectionnumberless{Solved exercises}