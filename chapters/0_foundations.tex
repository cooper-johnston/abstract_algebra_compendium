\chapter{Foundations}

\section{Prerequisites, conventions, and notation}

We will assume the reader is familiar with the concept of a set, set-builder notation, and basic set operations. By convention, the set of natural numbers $ \mathbb{N} $ will be taken to start from $ 1 $.

\section{Sets and relations}

\begin{defn}
For two sets $ A $ and $ B $, any subset of $ A\times B $ is called a \defnem{relation}, and for all $ (a,b) $ in this relation, we say $ a $ is \defnem{related to} $ b $, denoted, for example, by $ a\sim b $.
\end{defn}

\begin{defn}\label{defn:equiv_relation}
A relation $ a\sim b $ is called an \defnem{equivalence relation} if it is
\begin{enumerate}
    \item reflexive: for every $ a $, we have $ a\sim a $;
    \item symmetric: for every $ a,b $ such that $ a\sim b $, we have $ b\sim a $; and
    \item transitive: for every $ a,b,c $ such that $ a\sim b $ and $ b\sim c $, we have $ a\sim c $.
\end{enumerate}
\end{defn}

\begin{defn}
The set $ [a]=\{b\mid a\sim b\} $ is called the \defnem{equivalence class} of $ a $.
\end{defn}

\begin{thm}
Let $ \sim $ be an equivalence relation on a set $ X $. Then, the equivalence classes are disjoint and form a partition of $ X $.
\end{thm}
\begin{proof}
Let $ x_1,x_2\in X $ and consider the equivalence classes $ [x_1] $ and $ [x_2] $. Suppose they are not disjoint. Then, there exists a $ y $ such that $ y\in[x_1]\cap[x_2] $, so $ x_1\sim y $ and $ x_2\sim y $. By the symmetric property, $ x_1\sim y $ and $ y\sim x_2 $, so by the transitive property, $ x_1\sim x_2 $.

Now let $ x\in[x_1] $. Then, $ x_1\sim x $, and since $ x_1\sim x_2 $, we have $ x_2\sim x $, so $ x\in[x_2] $. Thus, $ [x_1]\subseteq[x_2] $, and similarly, $ [x_2]\subseteq[x_1] $, so $ [x_1]=[x_2] $.
\end{proof}

\section{Examples of proofs}

\begin{claim}[For a direct proof]
The product of two odd numbers is odd.
\end{claim}
\begin{proof}
Let $ a $ and $ b $ be odd. Then, $ a=2n+1 $ and $ b=2k+1 $ for some $ n,k\in\mathbb{Z} $, so we have
\begin{equation*}
    ab=(2n+1)(2k+1)=4nk+2n+2k+1=2(2nk+n+k)+1
\end{equation*}
which is odd since $ 2nk+n+k\in\mathbb{Z} $.
\end{proof}

\begin{claim}[For a proof by contraposition]
Let $ n\in\mathbb{Z} $. If $ n^2 $ is odd, then $ n $ is odd.
\end{claim}
\begin{proof}
Suppose $ n $ is even. Then, $ n=2k $ for some $ k\in\mathbb{Z} $, so
\begin{equation*}
    n^2=(2k)^2=4k^2=2(2k^2)
\end{equation*}
which is even since $ 2k^2\in\mathbb{Z} $. Hence, if $ n^2 $ is odd, then $ n $ is odd.
\end{proof}

\begin{claim}[For a proof by contradiction]
Let $ p\in\mathbb{Z} $. If $ p $ is prime, then $ \sqrt{p}\notin\mathbb{Q} $.
\end{claim}
\begin{proof}
Suppose $ \sqrt{p}\in\mathbb{Q} $. Then, there exist some $ a,b\in\mathbb{Z} $, $ b\neq 0 $ such that $ \sqrt{p}=a/b $. Without loss of generality, assume $ \gcd(a,b)=1 $. We see
\begin{equation*}
    p=\left(\frac{a}{b}\right)^2=\frac{a^2}{b^2} \iff pb^2=a^2 \implies p\mid a^2,
\end{equation*}
and since $ p $ is prime, we see $ p\mid a $. There must then exist some $ n\in\mathbb{Z} $ such that $ a=np $, so
\begin{equation*}
    pb^2=a^2=(np)^2=n^2p^2 \iff b^2=n^2p \implies p\mid b^2 \iff p\mid b.
\end{equation*}
Thus, $ p $ divides both $ a $ and $ b $, but this is a contradiction since $ \gcd(a,b)=1 $. Hence, $ \sqrt{p}\notin\mathbb{Q} $.
\end{proof}

\begin{claim}[For a proof by induction]
Let $ n\in\mathbb{N} $. If $ n\geq 5 $, then $ n!\geq 2^n $.
\end{claim}
\begin{proof}
For our base step, we see $ 5!=120 $ and $ 2^5=32 $, so $ 5!\geq 2^5 $.

As our inductive hypothesis, assume $ k!\geq 2^k $ for some $ k\geq 5 $. Then,
\begin{equation*}
    (k+1)k!\geq (k+1)2^k \geq 6\cdot 2^k \geq 2\cdot 2^k=2^{k+1} \implies (k+1)!\geq 2^{k+1}.
\end{equation*}
Hence, $ n!\geq 2^n $ for all $ n\geq 5 $.
\end{proof}

Note that this does not address the fact that $ 4!\geq 2^4$.

\sectionnumberless{Solved exercises}

\subsection*{Set operations}

For each of the following, find $ A\cap B $, $ A\cup B $, $ A\setminus B $, $ B\setminus A $, $ A\times B $, and $ B\times A $.

\begin{exer}
Let $ A=\{-1,1\} $ and $ B=\{1,2,3\} $.
\end{exer}
\begin{sltn}
We have
\begin{align*}
    A\cap B &= \{1\}, \\
    A\cup B &= \{-1,1,2,3\}, \\
    A\setminus B &= \{-1\}, \\
    B\setminus A &= \{2,3\}, \\
    A\times B &= \{(-1,1),(-1,2),(-1,3),(1,1),(1,2),(1,3)\}, \\
    B\times A &= \{(1,-1),(1,1),(2,-1),(2,1),(3,-1),(3,1)\}.\qedhere
\end{align*}
\end{sltn}

\begin{exer}
Let $ A=[-1,1] $ and $ B=(0,3] $.
\end{exer}
\begin{sltn}
We have
\begin{align*}
    A\cap B &= (0,1], \\
    A\cup B &= [-1,3], \\
    A\setminus B &= [-1,0], \\
    B\setminus A &= (1,3], \\
    A\times B &= \{(a,b)\mid a\in[-1,1],b\in (0,3]\}, \\
    B\times A &= \{(b,a)\mid b\in (0,3],a\in [-1,1]\}.\qedhere
\end{align*}
\end{sltn}

\begin{exer}
Let $ A=(1,3) $ and $ B=[0,\infty ) $.
\end{exer}
\begin{sltn}
We have
\begin{align*}
    A\cap B &= (1,3), \\
    A\cup B &= [0,\infty ), \\
    A\setminus B &= \varnothing, \\
    B\setminus A &= [0,1]\cup [3,\infty ), \\
    A\times B &= \{(a,b)\mid a\in(1,3),b\in [0,\infty )\}, \\
    B\times A &= \{(b,a)\mid b\in [0,\infty ),a\in (1,3)\}.\qedhere
\end{align*}
\end{sltn}

\subsection*{Proofs}

Let $ a,b,c\in\mathbb{N} $ where $ a $ and $ b $ are coprime. Prove the following.

\begin{exer}
If $ a\mid bc $, then $ a\mid c $.
\end{exer}
\begin{sltn}
Suppose $ a\mid bc $. Then, there exists some $ n\in\mathbb{Z} $ such that $ na=bc $, so $ b\mid na $. Now suppose $ n $ is not a multiple of $ b $. Then, $ a $ and $ b $ must share a common factor greater than 1, but $ a $ and $ b $ are coprime, so this is impossible. Therefore, $ n $ must be a multiple of $ b $; that is, there exists some $ k\in\mathbb{Z} $ such that $ n=kb $, so
\begin{equation*}
    na=bc \iff \frac{n}{b}a=c \iff \frac{bk}{b}a=c \iff ka=c \implies a\mid c.\qedhere
\end{equation*}
\end{sltn}

\begin{exer}
If $ a\mid c $ and $ b\mid c $, then $ ab\mid c $.
\end{exer}
\begin{sltn}
Suppose $ a\mid c $ and $ b\mid c $. Then, $ c $ is a multiple of $ a $, and $ c $ is a multiple of $ b $. Let $ p_1p_2\cdots p_n $ be the prime factorization of $ a $, and let $ q_1q_2\cdots q_k $ be the prime factorization of $ b $. Since $ a $ and $ b $ are coprime, we see $ \{p_1,p_2,\ldots,p_n\}\cap\{q_1,q_2,\ldots,q_k\}=\varnothing $, so the prime factorization of $ c $ must include all of the $ p_i $s and all of the $ q_i $s. Therefore, $ c $ is a multiple of $ p_1p_2\cdots p_nq_1q_2\cdots q_k=ab $, so $ ab\mid c $.
\end{sltn}