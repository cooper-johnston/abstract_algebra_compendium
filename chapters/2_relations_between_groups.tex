\chapter{Relations Between Groups}

\section{Group homomorphisms}

\begin{defn}
Let $ (G,\odot) $ and $ (G',\oslash) $ be groups. A mapping $ \phi:G\to G' $ is called a \defnem{group homomorphism} if for every $ x,y\in G $, we have
\begin{equation*}
    \phi(x\odot y)=\phi(x)\oslash\phi(y).
\end{equation*}
\end{defn}

\begin{defn}
A group homomorphism is called an \defnem{isomorphism} if it is a bijection. A group $ G $ is called \defnem{isomorphic to} a group $ G' $ if there exists an isomorphism $ \phi:G\to G' $. We denote this by $ G\simeq G' $.
\end{defn}

\begin{prop}
Let $ \phi:(G,\odot)\to(G',\oslash) $ be a homomorphism. Then,
\begin{enumerate}
    \item $ \phi(e_G)=e_{G'} $; and
    \item for all $ g\in G $, we have $ \phi(g^{-1})=(\phi(g))^{-1} $.
\end{enumerate}
\end{prop}
\begin{proof}~
\begin{enumerate}
    \item By definition, for all $ x\in G' $, we have $ x\oslash(x)^{-1}=(x)^{-1}\oslash x=e_{G'} $. In particular,
    \begin{equation*}
        e_{G'}=\phi(e_G)\oslash(\phi(e_G))^{-1}=(\phi(e_G))^{-1}\oslash\phi(e_G).
    \end{equation*}
    Since $ \phi $ is a homomorphism, we also have
    \begin{align*}
        \phi(e_G) &= \phi(e_G\odot e_G) \\
        \phi(e_G) &= \phi(e_G)\oslash\phi(e_G) \\
        \phi(e_G)\oslash(\phi(e_G))^{-1} &= \phi(e_G)\oslash\phi(e_G)\oslash(\phi(e_G))^{-1} \\
        e_{G'} &= \phi(e_G)\oslash e_{G'} \\
        e_{G'} &= \phi(e_G).
    \end{align*}

    \item By definition, $ (\phi(g))^{-1} $ is the inverse of $ \phi(g) $ in $ G' $. We see
    \begin{equation*}
        \phi(g^{-1})\oslash\phi(g)=\phi(g^{-1}\odot g)=\phi(e_G)=e_G',
    \end{equation*}
    so $ \phi(g^{-1}) $ is also the inverse of $ \phi(g) $ in $ G' $. Hence, by uniqueness of the inverse,
    \begin{equation*}
        \phi(g^{-1})=(\phi(g))^{-1}. \qedhere
    \end{equation*}
\end{enumerate}
\end{proof}

\begin{defn}
Let $ \phi:G\to G' $ be a homomorphism. The set 
\begin{equation*}
    \im(\phi)=\{\phi(g)\mid g\in G\}
\end{equation*}
is called the \defnem{image} of $ \phi $.
\end{defn}

\begin{prop}\label{prop:image_subgroup}
Let $ \phi:G\to G' $ be a homomorphism. Then, $ \im(\phi) $ is a subgroup of $ G' $.
\end{prop}
\begin{proof}
Let $ x,y\in\im(\phi) $. Then, there exist some $ u,v\in G $ such that $ \phi(u)=x $ and $ \phi(v)=y $, so
\begin{equation*}
    xy^{-1}=\phi(u)(\phi(v))^{-1}=\phi(u)\,\phi(v^{-1})=\phi(uv^{-1}).
\end{equation*}
Since $ uv^{-1}\in G $, we see $ xy^{-1}\in\im(\phi) $. Hence, $ \im(\phi) $ is a subgroup of $ G' $.
\end{proof}

\begin{defn}
Let $ \phi:G\to G' $ be a homomorphism. The set
\begin{equation*}
    \ker(\phi)=\{g\in G\mid\phi(g)=e_{G'}\}
\end{equation*}
is called the \defnem{kernel} of $ \phi $.
\end{defn}

\begin{thm}\label{thm:monomorphism}
Let $ \phi:G\to G' $ be a homomorphism. Then, $ \phi $ is injective if and only if $ \ker(\phi)=\{e_G\} $.
\end{thm}
\begin{proof}~
\begin{enumerate}
    \item[($ \Rightarrow $)] Suppose $ \phi $ is injective. Since $ \phi(e_G)=e_{G'} $, we know $ \{e_G\}\subseteq\ker(\phi) $. Let $ x\in\ker(\phi) $. Then, $ \phi(x)=e_{G'}=\phi(e_G) $, so since $ \phi $ is injective, $ x=e_G $, which implies $ \ker(\phi)\subseteq\{e_G\} $. Hence, $ \{e_G\}=\ker(\phi) $.

    \item[($ \Leftarrow $)] Suppose $ \ker(\phi)=\{e_G\} $. Let $ x,y\in G $ such that $ \phi(x)=\phi(y) $. Then,
    \begin{equation*}
        e_{G'}=\phi(x)(\phi(x))^{-1}=\phi(y)(\phi(x))^{-1}=\phi(y)\,\phi(x^{-1})=\phi(yx^{-1}).
    \end{equation*}
    Thus, $ yx^{-1}\in\ker(\phi) $, so $ yx^{-1}=e_G $, which implies $ y=x $. Hence, $ \phi $ is injective.\qedhere
\end{enumerate}
\end{proof}

\begin{thm}\label{thm:ker_normal_subgroup}
Let $ \phi:G\to G' $ be a homomorphism. Then, $ \ker(\phi) $ is a normal subgroup of $ G $.
\end{thm}
\begin{proof}
Let $ g\in G $ and $ x\in\ker(\phi) $. Then, $ \phi(x)=e_{G'} $, so
\begin{equation*}
    \phi(gxg^{-1})=\phi(g)\phi(x)\phi(g^{-1})=\phi(g)e_{G'}(\phi(g))^{-1}=\phi(g)(\phi(g))^{-1}=e_{G'}.\qedhere
\end{equation*}
\end{proof}

\begin{thm}
Let $ G $ be a group and $ H $ be a subgroup of $ G $. Then, $ H $ is a normal subgroup of $ G $ if and only if there exists a surjective homomorphism $ \phi:G\to G' $ for some group $ G' $ such that $ H=\ker(\phi) $.
\end{thm}
\begin{proof}Suppose $ H $ is a normal subgroup of $ G $. Consider the mapping
\begin{equation*}
    \begin{array}{rccc}
        \phi: & G & \to & G/H \\
        & g & \mapsto & gH
    \end{array}
\end{equation*}
where $ G/H $ has group structure as given in Theorem \ref{thm:cosets_group_structure}. Let $ x,y\in G $. We see
\begin{equation*}
    \phi(xy)=(xy)H=xHyH=\phi(x)\phi(y),
\end{equation*}
so $ \phi $ is a homomorphism, surjective by construction. Now let $ k\in\ker(\phi) $. Since $ H $ is the neutral element of $ G/H $, this means $ \phi(k)=kH=H $, which is true if and only if $ k\in H $. Hence, $ \ker(\phi)=H $. The converse is a direct consequence of Theorem \ref{thm:ker_normal_subgroup}.
\end{proof}

\begin{thm}
Let $ \phi:G\to G' $ be an isomorphism. Then, $ \phi^{-1} $ is an isomorphism.
\end{thm}
\begin{proof}
Let $ \odot $ denote the law of composition for group $ G $ and $ \oslash $ denote the law for $ G' $, let $ f=\phi^{-1} $, and let $ x,y\in G' $.  $ f $ is clearly well-defined, and we see
\begin{equation*}
    \phi(f(x)\odot f(y))=\phi(f(x))\oslash\phi(f(y))=x\oslash y=\phi(f(x\oslash y)).
\end{equation*}
Since $ \phi $ is injective, this implies $ f(x)\odot f(y)=f(x\oslash y) $, so $ f $ is a homomorphism. Injectivity and surjectivity can be easily verified. Hence, $ f $ is an isomorphism.
\end{proof}

\begin{thm}[Fundamental theorem on homomorphisms]
Let $ \phi:G\to G' $ be a homomorphism. Then, the mapping
\begin{equation*}
    \begin{array}{rccc}
        \psi: & G/\ker(\phi) & \to & \im(\phi) \\
        & g\ker(\phi) & \mapsto & \phi(g)
    \end{array}
\end{equation*}
is an isomorphism.
\end{thm}
\begin{proof}
We have four criteria for $ \psi $ to be an isomorphism:
\begin{enumerate}
    \item Let $ g,h $ be such that $ g\ker(\phi)=h\ker(\phi) $. Then, $ h^{-1}g\in\ker(\phi) $, so
    \begin{align*}
        \phi(h^{-1}g) &= e_{G'} \\
        (\phi(h))^{-1}\phi(g) &= e_{G'} \\
        \phi(g) &= \phi(h).
    \end{align*}
    Thus, $ \psi $ is well-defined.

    \item Let $ g\ker(\phi),h\ker(\phi)\in G/\ker(\phi) $. Then,
    \begin{align*}
        \psi(g\ker(\phi)\,h\ker(\phi)) &= \psi((gh)\ker(\phi))=\phi(gh)=\phi(g)\,\phi(h) \\
        &= \psi(g\ker(\phi))\,\psi(h\ker(\phi)),
    \end{align*}
    so $ \psi $ is a homomorphism.

    \item Let $ g\ker(\phi)\in\ker(\psi) $. Then, $ \psi(g\ker(\phi))=e_{G'} $, so $ g\in\ker(\phi) $, which implies $ g\ker(\phi)=\ker(\phi) $. Thus, by Theorem \ref{thm:monomorphism}, $ \psi $ is injective.

    \item $ \psi $ is surjective by construction since it maps to $ \im(\phi) $.\qedhere
\end{enumerate}
\end{proof}

This theorem is also known as the first isomorphism theorem.

\section{Permutation groups}

\begin{prop}\label{prop:permutations}
Let $ X $ be a set, and let $ \mathcal{S}(X) $ be the set of all bijections from $ X $ to $ X $. Then, $ (\mathcal{S}(X),\circ) $, where $ \circ $ is composition of mappings, is a group.
\end{prop}
\begin{proof}
We have three criteria for $ (\mathcal{S}(X),\circ) $ to be a group:
\begin{enumerate}
    \item Let $ \sigma,\tau\in\mathcal{S}(X) $. Then, $ \sigma\circ\tau $ is a mapping from $ X $ to $ X $. Let $ x,y\in X $ such that $ (\sigma\circ\tau)(x)=(\sigma\circ\tau)(y) $. Then, since $ \sigma $ and $ \tau $ are injective, we have
    \begin{align*}
        \sigma(\tau(x)) &= \sigma(\tau(y)) \\
        \tau(x) &= \tau(y) \\
        x &= y,
    \end{align*}
    so $ \sigma\circ\tau $ is injective, and any injective mapping from a set to itself is also surjective. Thus, $ \sigma\circ\tau\in\mathcal{S}(X) $, and we know composition of mappings is associative.

    \item The neutral element is naturally the identity mapping $ \id $:
    \begin{equation*}
        (\sigma\circ\id)(x)=\sigma(\id(x))=\sigma(x)=\id(\sigma(x))=(\id\circ\,\sigma)(x).
    \end{equation*}

    \item Since every $ \sigma\in\mathcal{S}(X) $ is injective, every $ \sigma $ has an inverse mapping.\qedhere
\end{enumerate}
\end{proof}

\begin{defn}
Take $ \mathcal{S}(X) $ as defined in Proposition \ref{prop:permutations} for some set $ X $. A subgroup of $ \mathcal{S}(X) $ is called a \defnem{permutation group}. Any mapping in such a group is called a \defnem{permutation}.
\end{defn}

\begin{thm}[Cayley's theorem]
Every group is isomorphic to a permutation group.
\end{thm}
\begin{proof}
Let $ G $ be a group. For each $ a\in G $, we define a mapping
\begin{equation*}
    \begin{array}{rccc}
        \sigma_a: & G & \to & G \\
        & g & \mapsto & ag
    \end{array}.
\end{equation*}
For some $ b\in G $, let $ x,y\in G $ such that $ \sigma_b(x)=\sigma_b(y) $. Then, $ bx=by $, so left cancellation implies $ x=y $. Thus, $ \sigma_b $ is injective, and any injective mapping from a set to itself is also surjective, so $ \sigma_b\in\mathcal{S}(G) $.

Now, we define a mapping
\begin{equation*}
    \begin{array}{rccc}
        \phi: & G & \to & \mathcal{S}(G) \\
        & g & \mapsto & \sigma_g
    \end{array}.
\end{equation*}
Let $ a,b\in G $. Then, for all $ x\in G $, we have
\begin{equation*}
    \phi(ab)(x)=\sigma_{ab}(x)=abx=a\sigma_b(x)=\sigma_a(\sigma_b(x))=\phi(a)\circ\phi(b),
\end{equation*}
so $ \phi $ is a homomorphism. If $ a\in\ker(\phi) $, then $ \phi(a)=\sigma_a=\id $, which is true if and only if for all $ x\in G $, we have
\begin{equation*}
    \phi(a)(x)=\sigma_a(x)=ax=x \iff a=e_G.
\end{equation*}
Thus, $ \ker(\phi)=\{e_G\} $, so $ \phi $ is injective. By Proposition \ref{prop:image_subgroup}, $ \im(\phi) $ is a subgroup of $ \mathcal{S}(X) $; hence, we can construct an isomorphism $ \psi:G\to\im(\phi) $.
\end{proof}

\begin{defn}\label{defn:symmetric_group}
Let $ A=\{1,2,\ldots,n\} $ for some $ n\in\mathbb{N} $. Then, $ \mathcal{S}_n=\mathcal{S}(A) $ is called the \defnem{symmetric group} on $ n $ elements.
\end{defn}

More generally, $ \mathcal{S}_n $ can be used to describe the group of permutations of any finite set. Since any finite set is isomorphic to a subset of $ \mathbb{N} $, we can apply this definition by assigning a label in $ A $ to each element. The results we will show for $ \mathcal{S}_n $ therefore apply with this generalization as well.

Note that for any $ n\in\mathbb{N} $, we have $ \lvert\mathcal{S}_n\rvert=n! $. This may be familiar if you recall the notion of a permutation of a set as a rearrangement of its elements. The notation may also be familiar\textemdash consider the following permutation $ \sigma\in\mathcal{S}_5 $:
\begin{align*}
    1 &\mapsto 3 \\
    2 &\mapsto 2 \\
    3 &\mapsto 5 \\
    4 &\mapsto 4 \\
    5 &\mapsto 1.
\end{align*}
This can be written as
\begin{equation*}
    \sigma=
    \begin{pmatrix}
        1 & 2 & 3 & 4 & 5 \\
        3 & 2 & 5 & 4 & 1
    \end{pmatrix}.
\end{equation*}

\begin{defn}
Let $ \sigma=\mathcal{S}_n $. The set
\begin{equation*}
    \supp(\sigma)=\{i\in\{1,2,\ldots,n\}\mid\sigma(i)\neq i\}
\end{equation*}
is called the \defnem{support} of $ \sigma $.
\end{defn}

\begin{prop}
Let $ \sigma,\tau\in\mathcal{S}_n $. If $ \supp(\sigma)\cap\supp(\tau)=\varnothing $, then $ \sigma\circ\tau=\tau\circ\sigma $.
\end{prop}
\begin{proof}
Let $ i\in\{1,2,\ldots,n\} $. We have three cases:
\begin{enumerate}
    \item Suppose $ i\notin\supp(\sigma)\cup\supp(\tau) $. Then, $ \sigma(i)=\tau(i)=i $, so
    \begin{align*}
        (\sigma\circ\tau)(i) &= \sigma(\tau(i))=\sigma(i)=i=\tau(i)=\tau(\sigma(i))=(\tau\circ\sigma)(i).
    \end{align*}

    \item Suppose $ i\in\supp(\sigma) $. Then, $ i\notin\supp(\tau) $, so
    \begin{equation*}
        (\sigma\circ\tau)(i)=\sigma(\tau(i))=\sigma(i),
    \end{equation*}
    and since $ i\in\supp(\sigma) $, we have $ \sigma(i)\in\supp(\sigma) $, so $ \sigma(i)\notin\supp(\tau) $. Thus,
    \begin{equation*}
        (\tau\circ\sigma)(i)=\tau(\sigma(i))=\sigma(i)=(\sigma\circ\tau)(i).
    \end{equation*}

    \item If $ i\in\supp(\tau) $, the proof can be done in the same way as in the above case.
\end{enumerate}
Hence, $ \sigma\circ\tau=\tau\circ\sigma $.
\end{proof}

\subsection*{Cycles}

\begin{defn}
An element $ \sigma\in\mathcal{S}_n $ is called a \defnem{cycle} if there exists some $ x\in\{1,2,\ldots,n\} $ such that $ \supp(\sigma)=\{\sigma^i(x)\mid i\in\mathbb{N}\} $. Let $ l=\lvert\supp(\sigma)\rvert $. We denote the cycle
\begin{equation*}
    \left(x,\sigma(x),\ldots,\sigma^{l-1}(x)\right)
\end{equation*}
where $ l $ is called its \defnem{length}. A cycle of length 2 is called a \defnem{transposition}.
\end{defn}

\begin{prop}
Let $ \sigma $ be a cycle of length $ l $. Then, $ \ord(\sigma)=l $.
\end{prop}

This follows by construction.

\begin{prop}\label{prop:orbits}
Let $ \sigma\in\mathcal{S}_n $, and let $ A=\{1,2,\ldots,n\} $. Then, the relation $ \sim $ on $ A $ defined such that for all $ a,b\in A $,
\begin{equation*}
    a\sim b \iff \text{there exists some }k\in\mathbb{Z}\text{ such that }b=\sigma^k(a)
\end{equation*}
is an equivalence relation.
\end{prop}
\begin{proof}
We have three criteria for $ \sim $ to be an equivalence relation:
\begin{enumerate}
    \item Since $ a=\sigma^0(a) $, we have $ a\sim a $ (reflexive).
    \item Suppose $ a\sim b $. Then, $ b=\sigma^k(a) $ for some $ k\in\mathbb{Z} $, so $ a=\sigma^{-k}(b) $. Thus, $ b\sim a $ (symmetric).
    \item Let $ c\in A $. Suppose $ a\sim b $ and $ b\sim c $. Then, $ b=\sigma^k(a) $ and $ c=\sigma^m(b) $ for some $ k,m\in\mathbb{Z} $, so $ c=\sigma^m(\sigma^k(a))=\sigma^{m+k}(a) $. Thus, $ a\sim c $ (transitive).\qedhere
\end{enumerate}
\end{proof}

\begin{cor}[Alternative definition of a cycle]
Take $ \sim $ as defined in Proposition \ref{prop:orbits} for some $ \sigma\in\mathcal{S}_n $. Then, $ \sigma $ is a cycle if and only if $ \sim $ has at most one equivalence class containing more than one element.
\end{cor}

\begin{thm}\label{thm:cycle_decomp}
Let $ \sigma\in\mathcal{S}_n $. Then, there exist some unique cycles $ \tau_1,\tau_2,\ldots,\tau_k $ with disjoint supports such that $ \sigma=\tau_1\circ\tau_2\circ\cdots\circ\tau_k $. In other words, every permutation of a finite set can be decomposed as the product of unique cycles with disjoint supports.
\end{thm}
\begin{proof}
Let $ A_1,A_2,\ldots,A_k $ be the equivalence classes of $ \sim $, and let $ \tau_1,\tau_2,\ldots,\tau_k $ be the cycles defined such that 
\begin{equation*}
    \tau_i(x)=
    \left\{\begin{array}{ll}
        \sigma(x), & x\in A_i \\
        x, & \text{otherwise.}
    \end{array}\right.
\end{equation*}
We see $ \sigma=\tau_1\circ\tau_2\circ\cdots\circ\tau_k $, and since $ A_1,A_2,\ldots,A_k $ are necessarily disjoint, $ \tau_1,\tau_2,\ldots,\tau_k $ have disjoint supports.
\end{proof}

\begin{defn}
Let $ \sigma\in\mathcal{S}_n $ with decomposition $ \sigma=\tau_1\circ\tau_2\circ\cdots\circ\tau_k $ as given by Theorem \ref{thm:cycle_decomp}. Let $ l_1,l_2,\ldots,l_k $ denote the lengths of $ \tau_1,\tau_2,\ldots,\tau_k $, respectively, where $ l_1\geq l_2\geq\cdots\geq l_k $. The sequence $ (l_1,l_2,\ldots,l_k) $ is called the \defnem{type} of $ \sigma $.
\end{defn}

\begin{prop}
Let $ \sigma\in\mathcal{S}_n $ with type $ (l_1,l_2,\ldots,l_k) $. Then,
\begin{equation*}
    \ord(\sigma)=\lcm\{l_1,l_2,\ldots,l_k\}.
\end{equation*}
\end{prop}
\begin{proof}
We can decompose $ \sigma $ into cycles as $ \sigma=\tau_1\circ\tau_2\circ\cdots\circ\tau_k $ where $ \tau_1,\tau_2,\ldots,\tau_k $ have length $ l_1,l_2,\ldots,l_k $, respectively. Since the $ \tau_i $s have disjoint supports, they commute, so for every $ m\in\mathbb{N} $, we have
\begin{equation*}
    \sigma^m=\tau_1^m\circ\tau_2^m\circ\cdots\circ\tau_k^m.
\end{equation*}
Since $ \ord(\tau_i)=l_i $ for $ 1\leq i\leq k $, we see that if $ \sigma^m=\id $, then $ m $ is a multiple of each of the $ l_i $s. Hence, by definition, $ \ord(\sigma) $ is the lowest such $ m $.
\end{proof}

\subsection*{Transpositions and alternating groups}

\begin{cor}[to Theorem \ref{thm:cycle_decomp}]
Every permutation in $ \mathcal{S}_n $ can be decomposed as the product of transpositions.
\end{cor}

\begin{prop}
Let $ \sigma\in\mathcal{S}_n $. Either all transposition decompositions of $ \sigma $ are the product of an even number of transpositions, or all of them are the product of an odd number of transpositions.
\end{prop}
\begin{proof}
Consider the group of permutations of the rows of the $ n\times n $ identity matrix $ I_n $. Let us call this group $ P $. As remarked following Definition \ref{defn:symmetric_group}, $ P\simeq\mathcal{S}_n $. We know $ \det(I_n)=1 $, and transposing any two rows of a square matrix changes the sign of its determinant.

Let $ \rho\in P $, and let $ A=\rho(I_n) $. Suppose $ \rho $ can be decomposed as an even number of transpositions. Then, $ \det(A)=1 $. Now suppose $ \rho $ can also be decomposed as an odd number of transpositions. Then, $ \det(A)=-1 $, a contradiction. Hence, no $ \rho\in P $ can be decomposed into the product of both an even number and an odd number of transpositions.
\end{proof}

\begin{defn}
Let $ \sigma\in\mathcal{S}_n $, and let $ k $ be the number of transpositions in some transposition decomposition of $ \sigma $. The number $ \epsilon(\sigma)=(-1)^k $ is called the \defnem{signature} of $ \sigma $. The permutation $ \sigma $ is called \defnem{even} if $ k $ is even or \defnem{odd} if $ k $ is odd.
\end{defn}

\begin{prop}\label{prop:alternating_group}
Let $ \mathcal{A}_n=\{\sigma\in\mathcal{S}_n\mid\epsilon(\sigma)=1\} $. Then, $ \mathcal{A}_n $ is a normal subgroup of $ \mathcal{S}_n $.
\end{prop}
\begin{proof}
Let $ \alpha\in\mathcal{A}_n $ and $ \sigma\in\mathcal{S}_n $. For some $ k,m\in\mathbb{N} $, $ \alpha $ can be decomposed as the product of some number $ 2k $ of transpositions and $ \sigma $ can be decomposed as the product of some number $ m $ of transpositions, so there exists a decomposition of $ \sigma\circ\alpha\circ\sigma^{-1} $ into some number $ m+2k+m=2(m+k) $ of transpositions. Since $ 2(m+k) $ is even, $ \sigma\circ\alpha\circ\sigma^{-1}\in\mathcal{A}_n $. Hence, by Theorem \ref{thm:normal_subgroup}, $ \mathcal{A}_n $ is a normal subgroup of $ \mathcal{S}_n $.
\end{proof}

We can alternatively show that the mapping
\begin{equation*}
    \begin{array}{rccc}
        \epsilon: & (\mathcal{S}_n,\circ) & \to & (\{-1,1\},\cdot) \\
        & \sigma & \mapsto & \epsilon(\sigma)
    \end{array}
\end{equation*}
is a group homomorphism and that $ \mathcal{A}_n=\ker(\epsilon) $. By Theorem \ref{thm:ker_normal_subgroup}, this implies $ \mathcal{A}_n $ is a normal subgroup of $ \mathcal{S}_n $.

\begin{defn}
$ \mathcal{A}_n $ as defined in Proposition \ref{prop:alternating_group} is called the \defnem{alternating group} on $ n $ elements.
\end{defn}

\section{Finitely generated abelian groups}

Recall the Cartesian product of two sets $ A $ and $ B $:
\begin{equation*}
    A\times B=\{(a,b)\mid a\in A,b\in B\}.
\end{equation*}
We can give group structure to the Cartesian product of an arbitrary number of groups.

\begin{prop}\label{prop:direct_product}
Let $ G_1 $ and $ G_2 $ be two groups. The set $ G_1\times G_2 $ together with the law of composition
\begin{equation*}
    \begin{array}{rccc}
        \odot: & (G_1\times G_2)\times(G_1\times G_2) & \to & G_1\times G_2 \\
        & ((a_1,a_2),(b_1,b_2)) & \mapsto & (a_1b_1,a_2b_2)
    \end{array}
\end{equation*}
is a group.
\end{prop}
\begin{proof}
We have three criteria for $ (G_1\times G_2,\odot) $ to be a group:
\begin{enumerate}
    \item Let $ (a_1,a_2),(b_1,b_2),(c_1,c_2)\in G_1\times G_2 $. Then,
    \begin{align*}
        ((a_1,a_2)\odot(b_1,b_2))\odot(c_1,c_2) &= (a_1b_1,a_2b_2)\odot(c_1,c_2) \\
        &= ((a_1b_1)c_1,(a_2b_2)c_2) \\
        &= (a_1(b_1c_1),a_2(b_2c_2)) \\
        &= (a_1,a_2)\odot(b_1c_1,b_2c_2) \\
        &= (a_1,a_2)\odot((b_1,b_2)\odot(c_1,c_2)),
    \end{align*}
    so $ \odot $ is associative.
    \item Let $ e_1 $ be the neutral element of $ G_1 $ and $ e_2 $ be the neutral element of $ G_2 $. Naturally, the neutral element of $ G_1\times G_2 $ is then $ (e_1,e_2) $:
    \begin{equation*}
        (a_1,a_2)\odot(e_1,e_2)=(a_1e_1,a_2e_2)=(a_1,a_2).
    \end{equation*}
    \item Naturally, the inverse of $ (a_1,a_2) $ is $ (a_1^{-1},a_2^{-1}) $:
    \begin{equation*}
        (a_1,a_2)\odot(a_1^{-1},a_2^{-1})=(a_1a_2^{-1},a_2a_2^{-1})=(e_1,e_2).\qedhere
    \end{equation*}
\end{enumerate}
\end{proof}

\begin{cor}\label{cor:direct_product}
Let $ \{G_i\}_{i\in I} $ be a family of groups for some non-empty (perhaps infinite) index set $ I $. The set
\begin{equation*}
    \prod_{i\in I}G_i=\{(g_i)_{i\in I}\mid g_i\in G_i\text{ for all }i\in I\},
\end{equation*}
where $ (g_i)_{i\in I} $ denotes the sequence of $ g_i $s as a touple, together with the law of composition $ \odot $ defined such that for all $ (g_i)_{i\in I},(h_i)_{i\in I}\in\{G_i\}_{i\in I} $, we have
\begin{equation*}
    (g_i)_{i\in I}\odot(h_i)_{i\in I}=(g_ih_i)_{i\in I}
\end{equation*}
is a group.
\end{cor}

\begin{cor}\label{cor:direct_sum}
Let $ \{G_i\}_{i\in I} $ be a family of abelian groups for some non-empty index set $ I $. The set
\begin{equation*}
    \bigoplus_{i\in I}G_i=\{(g_i)_{i\in I}\mid g_i\in G_i\text{ for all }i\in I, g_i\neq e_i\text{ for only finitely many }i\in I\},
\end{equation*}
where $ e_i $ denotes the neutral element of group $ G_i $, together with the law of composition $ \odot $ given in Corollary \ref{cor:direct_product}, is a group. Furthermore, $ \bigoplus_{i\in I}G_i=\prod_{i\in I}G_i $ when $ I $ is finite; otherwise, $ \bigoplus_{i\in I}G_i $ is a proper subgroup of $ \prod_{i\in I}G_i $.
\end{cor}

\begin{defn}
Let $ \{G_i\}_{i\in I} $ be a family of groups for some non-empty index set $ I $. The group $ \prod_{i\in I}G_i $ from Corollary \ref{cor:direct_product} is called the \defnem{direct product} of the $ G_i $s.

If the $ G_i $s are abelian, the group $ \bigoplus_{i\in I}G_i $ from Corollary \ref{cor:direct_sum} is called the \defnem{direct sum} of the $ G_i $s.
\end{defn}

For a finite family of groups $ \{G_1,G_2,\ldots,G_n\} $, we can denote their direct sum as
\begin{equation*}
    G_1\oplus G_2\oplus\cdots\oplus G_n.
\end{equation*}

\section{Group action on a set}

\section{Sylow's theorem (?)}

\sectionnumberless{Solved exercises}

\subsection*{Group homomorphisms}

We define the mapping
\begin{equation*}
    \begin{array}{rrcl}
        f: & (\mathbb{R},+) & \to & (\mathbb{C}\setminus\{0\},\cdot) \\
        & x & \mapsto & e^{i(2\pi x)} 
    \end{array}
\end{equation*}
where $ (\mathbb{R},+) $ and $ (\mathbb{C}\setminus\{0\},\cdot) $ are presumed to be groups (this can be shown).

\begin{exer}
Show that $ f $ is a group homomorphism.
\end{exer}
\begin{sltn}
Let $ x,y\in\mathbb{R} $. Then,
\begin{equation*}
    f(x+y)=e^{i(2\pi(x+y))}=e^{i(2\pi x)+i(2\pi y)}=e^{i(2\pi x)}\cdot e^{i(2\pi y)}=f(x)\cdot f(y)
\end{equation*}
so $ f $ is a group homomorphism.
\end{sltn}

\begin{exer}
Find $ \ker(f) $.
\end{exer}
\begin{sltn}
We know $ e_{\mathbb{C}}=1 $, so
\begin{equation*}
    \mathrm{ker}(f)=\{x\in\mathbb{R}\mid e^{i(2\pi x)}=1\}=\mathbb{Z}.\qedhere
\end{equation*}
\end{sltn}

\begin{exer}
Find $ \im(f) $.
\end{exer}
\begin{sltn}
By definition, we have
\begin{equation*}
    \mathrm{im}(f)=\{e^{i(2\pi x)}\in\mathbb{C}\setminus\{0\}\mid x\in\mathbb{R}\}=\{z\in\mathbb{C}\mid\lvert z\rvert=1\},
\end{equation*}
which is the complex unit circle group, often denoted $ \mathbb{T} $.
\end{sltn}

\begin{exer}
Construct an isomorphism from $ f $ using the fundamental theorem on homomorphisms.
\end{exer}
\begin{sltn}
We define
\begin{equation*}
    \begin{array}{rrcl}
        \psi: & (\mathbb{R},+)/\mathrm{ker}(f) & \to & \mathrm{im}(f) \\
        & x\mathrm{ker}(f) & \mapsto & f(x) 
    \end{array}
    \equiv
    \begin{array}{rcl}
        (\mathbb{R},+)/\mathbb{Z} & \to & \mathbb{T} \\
        x\mathbb{Z} & \mapsto & e^{i(2\pi x)}
    \end{array}.\qedhere
\end{equation*}
\end{sltn}

\subsection*{Isomorphisms}

Let $ G $ be a group. For each $ a\in G $, we define the mapping
\begin{equation*}
    \begin{array}{rrcl}
        f_a: & G & \to & G \\
        & x & \mapsto & axa^{-1} 
    \end{array}.
\end{equation*}

\begin{exer}
Show that $ f_a $ is an isomorphism.
\end{exer}
\begin{sltn}
Let $ x,y\in G $. Then,
\begin{equation*}
    f_a(xy)=a(xy)a^{-1}=ax(a^{-1}a)ya^{-1}=(axa^{-1})(aya^{-1})=f_a(x)f_a(y),
\end{equation*}
so $ f_a $ is a group homomorphism. Additionally, for every $ z $ in the codomain $ G $, we have $ z=f_a(a^{-1}za) $, so $ f_a $ is surjective, and any surjective mapping between two sets of the same cardinality is also injective. Hence, $ f_a $ is an isomorphism.
\end{sltn}

\begin{exer}
Show that for all $ x\in G $, we have $ \ord(f_a(x))=\ord(x) $.
\end{exer}
\begin{sltn}
Let $ n=\mathrm{ord}(x) $. Then, $ n $ is the minimal positive integer such that
\begin{align*}
    x^n &= e_G \\
    ax^na^{-1} &= ae_Ga^{-1} \\
    ax^na^{-1} &= aa^{-1} \\
    f_a(x^n) &= e_G.
\end{align*}
Since $ f_a $ is a homomorphism, $ f_a(x^n)=(f_a(x))^n $. Hence, $ \ord(f_a(x))=n $.
\end{sltn}

\subsection*{The dihedral group}

Let $ p $ be a prime number greater than or equal to $ 3 $, and let $ G $ be a group of cardinality $ 2p $.

\begin{exer}
What can we say if $ G $ has an element of order $ 2p $?
\end{exer}
\begin{sltn}
Let $ g\in G $ such that $ \ord(g)=2p $. Then, since $ \langle g\rangle\subseteq G $ and
\begin{equation*}
    \lvert\langle g\rangle\rvert=\ord(g)=2p=\lvert G\rvert,
    \end{equation*}
we see $ G=\langle g\rangle $. Hence, $ G $ is a cyclic group.
\end{sltn}

Now assume $ G $ has no element of order $ 2p $.

\begin{exer}
Show that $ G $ has an element of order $ p $.
\end{exer}
\begin{sltn}
Let $ H $ be a subgroup of $ G $. By Lagrange's theorem, $ \lvert H\rvert $ divides $ \lvert G\rvert $, so $ \lvert H\rvert\in\{1,2,p,2p\} $. Let $ g\in G\setminus\{e\} $. Since $ \langle g\rangle $ is a subgroup of $ G $ and since, by assumption, $ \ord(g)\neq 2p $, we see $ \ord(g)\in\{2,p\} $.

Suppose $ G $ does not have an element of order $ p $. Then, for all $ x,y\in G\setminus\{e\} $, we have $ \ord(x)=\ord(y)=2 $, so $ x^2=y^2=e $. Thus,
\begin{align*}
    (xy)(xy) &= e \\
    xyxyy &= ey \\
    xyx &= y \\
    xxyx &= xy \\
    yx &= xy,
\end{align*}
so $ G $ is abelian. We therefore have that $ \{e,x,y,xy\} $ is a subgroup of $ G $ of order 4, a contradiction. Hence, $ G $ has an element of order $ p $.
\end{sltn}

\begin{exer}
Let $ a\in G $ such that $ \ord(a)=p $, let $ H $ be the subgroup of $ G $ generated by $ a $, and let $ b\in G\setminus H $. Show that
\begin{equation*}
    G=\{e,a,a^2,\ldots,a^{p-1},b,ba,ba^2,\ldots,ba^{p-1}\}.
\end{equation*}
\end{exer}
\begin{sltn}
Since $ b\notin H $, we see
\begin{equation*}
    H=\langle a\rangle=\{e,a,a^2,\ldots,a^{p-1}\} \quad\text{and}\quad bH=\{b,ba,ba^2,\ldots,ba^{p-1}\}
\end{equation*}
are cosets of $ H $ in $ G $, each of cardinality $ p $. Since the cosets are disjoint, we have $ \lvert H\cup bH\rvert=2p=\lvert G\rvert $, so
\begin{equation*}
    G=H\cup bH=\{e,a,a^2,\ldots,a^{p-1},b,ba,ba^2,\ldots,ba^{p-1}\}.\qedhere
\end{equation*}
\end{sltn}

\begin{exer}
Show that $ b^2=e $.
\end{exer}
\begin{sltn}
We have shown that for every $ g\in G $, we have $ g\in H $ or $ g\in bH $. Since $ b\notin H $, there does not exist any $ n\in\mathbb{Z} $ such that $ a^n=b $. Suppose $ b^2\in bH $. Then, there exists some $ n\in\mathbb{Z} $ such that
\begin{align*}
    b^2 &= ba^n \\
    bb &= ba^n \\
    b^{-1}bb &= b^{-1}ba^n \\
    b &= a^n,
\end{align*}
a contradiction. Thus, $ b^2\in H $, so there exists some $ m\in\mathbb{Z} $ such that $ b^2=a^m $.

We have also shown that for every $ g\in G\setminus\{e\} $, we have $ \ord(g)=2 $ or $ \ord(g)=p $. Suppose $ \ord(b)=p $. Then, $ b^p=e $. By Bézout's theorem, since $ \gcd(2,p)=1 $, there exist some $ k,l\in\mathbb{Z} $ such that $ 2k+pl=1 $. Thus,
\begin{equation*}
    b=b^1=b^{2k+pl}=b^{2k}\,b^{pl}=(b^2)^k(b^p)^l=(b^2)^k=(a^m)^l=a^{mk},
\end{equation*}
a contradiction. Hence, $ \ord(b)=2 $.
\end{sltn}

\begin{exer}
Show that $ ab=ba^{p-1} $.
\end{exer}
\begin{sltn}
We have shown that for every $ g\in G\setminus H $, we have $ g^2=e $. Thus,
\begin{align*}
    (ba^{p-1})^2 &= e \\
    ba^{p-1}ba^{p-1} &= e \\
    ba^{p-1}ba^p &= a \\
    ba^{p-1}be &= a \\
    ba^{p-1}bb &= ab \\
    ba^{p-1} &= ab.\qedhere
\end{align*}
\end{sltn}

\begin{exer}
We define the \defnem{dihedral group} $ \mathcal{D}_n $ as the group of symmetries of the regular $ n $-gon, consisting of $ n $ rotations of angle $ 2\pi k/n $ for $ k\in\{0,1,\ldots,n-1\} $ and $ n $ reflections about the lines intersecting its center and each vertex. It can be shown that for any rotation $ r\in\mathcal{D}_n $ and any reflection $ s\in\mathcal{D}_n $, $ r $ and $ s $ generate $ \mathcal{D}_n $; that is,
\begin{equation*}
    \mathcal{D}_n=\{\mathrm{id},r,r^2,\ldots,r^{n-1},sr,sr^2,\ldots,sr^{n-1}\}.
\end{equation*}
Show that $ G\simeq\mathcal{D}_p $.
\end{exer}
\begin{sltn}
Let $ r $ be a rotation in $ \mathcal{D}_p $, and let $ s $ be a reflection in $ \mathcal{D}_p $. Consider the mapping $ \phi:G\to\mathcal{D}_p $ such that for all $ n\in\mathbb{Z} $,
\begin{align*}
    \phi(a^n) &= r^n, & \phi(ba^n) &= sr^n.
\end{align*}
Let $ n,m\in\mathbb{Z} $. Any composition of elements in $ G $ is of one of the following forms:
\begin{align*}
    a^na^m &= a^{n+m}, \\
    a^nba^m &= (a^{n-1}a)ba^m=a^{n-1}(ba^{p-1})a^m=a^{n-1}ba^{p-1+m}=\cdots=ba^{n(p-1)+m}, \\
    ba^na^m &= ba^{n+m}, \\
    ba^nba^m &= b(ba^{n(p-1)+m})=a^{n(p-1)+m}.
\end{align*}
Thus, $ \phi $ is well-defined, and it can be shown through straightforward computations that for all $ x,y\in G $, we have $ \phi(xy)=\phi(x)\,\phi(y) $, so $ \phi $ is a homomorphism.
    
Since every element of $ G $ maps to a unique element in $ \mathcal{D}_p $, we see $ \phi $ is injective. Additionally, since
\begin{equation*}
    \mathcal{D}_p=\{\mathrm{id},r,r^2,\ldots,r^{p-1},sr,sr^2,\ldots,sr^{p-1}\},
\end{equation*}
we see each element of $ \mathcal{D}_p $ is reached by some element of $ G $, so $ \phi $ is surjective. Hence, $ \phi $ is an isomorphism.
\end{sltn}