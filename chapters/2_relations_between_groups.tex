\chapter{Relations Between Groups}

\section{Group homomorphisms}

\begin{defn}
Let $ (G,\odot) $ and $ (G',\oslash) $ be groups. A mapping $ \phi:G\to G' $ is called a \defnem{group homomorphism} if for every $ x,y\in G $, we have
\begin{equation*}
    \phi(x\odot y)=\phi(x)\oslash\phi(y).
\end{equation*}
\end{defn}

\begin{defn}
A group homomorphism is called an \defnem{isomorphism} if it is a bijection. A group $ G $ is called \defnem{isomorphic to} a group $ G' $ if there exists an isomorphism $ \phi:G\to G' $. We denote this by $ G\simeq G' $.
\end{defn}

\begin{prop}
Let $ \phi:(G,\odot)\to(G',\oslash) $ be a homomorphism. Then,
\begin{enumerate}
    \item $ \phi(e_G)=e_{G'} $; and
    \item for all $ g\in G $, we have $ \phi(g^{-1})=(\phi(g))^{-1} $.
\end{enumerate}
\end{prop}
\begin{proof}~
\begin{enumerate}
    \item \todo{Do this proof!}
    \item By definition, $ (\phi(g))^{-1} $ is the inverse of $ \phi(g) $ in $ G' $. We see
    \begin{equation*}
        \phi(g^{-1})\oslash\phi(g)=\phi(g^{-1}\odot g)=\phi(e_G)=e_G',
    \end{equation*}
    so $ \phi(g^{-1}) $ is also the inverse of $ \phi(g) $ in $ G' $. Hence, by uniqueness of the inverse,
    \begin{equation*}
        \phi(g^{-1})=(\phi(g))^{-1}. \qedhere
    \end{equation*}
\end{enumerate}
\end{proof}

\begin{defn}
Let $ \phi:G\to G' $ be a homomorphism. The set 
\begin{equation*}
    \im(\phi)=\{\phi(g)\mid g\in G\}
\end{equation*}
is called the \defnem{image} of $ \phi $.
\end{defn}

\begin{prop}
Let $ \phi:G\to G' $ be a homomorphism. Then, $ \im(\phi) $ is a subgroup of $ G' $.
\end{prop}
\begin{proof}
Let $ x,y\in\im(\phi) $. Then, there exist some $ u,v\in G $ such that $ \phi(u)=x $ and $ \phi(v)=y $, so
\begin{equation*}
    xy^{-1}=\phi(u)(\phi(v))^{-1}=\phi(u)\,\phi(v^{-1})=\phi(uv^{-1}).
\end{equation*}
Since $ uv^{-1}\in G $, we see $ xy^{-1}\in\im(\phi) $. Hence, $ \im(\phi) $ is a subgroup of $ G' $.
\end{proof}

\begin{defn}
Let $ \phi:G\to G' $ be a homomorphism. The set
\begin{equation*}
    \ker(\phi)=\{g\in G\mid\phi(g)=e_G'\}
\end{equation*}
is called the \defnem{kernel} of $ \phi $.
\end{defn}

\begin{thm}\label{thm:monomorphism}
Let $ \phi:G\to G' $ be a homomorphism. Then, $ \phi $ is injective if and only if $ \ker(\phi)=\{e_G\} $.
\end{thm}
\begin{proof}~
\begin{enumerate}
    \item[($ \Rightarrow $)] Suppose $ \phi $ is injective. Since $ \phi(e_G)=e_{G'} $, we know $ \{e_G\}\subseteq\ker(\phi) $. Let $ x\in\ker(\phi) $. Then, $ \phi(x)=e_{G'}=\phi(e_G) $, so since $ \phi $ is injective, $ x=e_G $. Hence, $ \{e_G\}=\ker(\phi) $.

    \item[($ \Leftarrow $)] Suppose $ \ker(\phi)=\{e_G\} $. Let $ x,y\in G $ such that $ \phi(x)=\phi(y) $. Then,
    \begin{equation*}
        e_{G'}=\phi(x)(\phi(x))^{-1}=\phi(y)(\phi(x))^{-1}=\phi(y)\,\phi(x^{-1})=\phi(yx^{-1}).
    \end{equation*}
    Thus, $ yx^{-1}\in\ker(\phi) $, so $ yx^{-1}=e_G $, which implies $ y=x $. Hence, $ \phi $ is injective.\qedhere
\end{enumerate}
\end{proof}

\begin{thm}\label{thm:ker_normal_subgroup}
Let $ \phi:G\to G' $ be a homomorphism. Then, $ \ker(\phi) $ is a normal subgroup of $ G $.
\end{thm}
\begin{proof}
\todo{Do this proof!}
\end{proof}

\begin{thm}
Let $ G $ be a group and $ H $ be a subgroup of $ G $. Then, $ H $ is a normal subgroup of $ G $ if and only if there exists a surjective homomorphism $ \phi:G\to G' $ for some group $ G' $ such that $ H=\ker(\phi) $.
\end{thm}
\begin{proof}
\todo{Do this proof!}
\end{proof}

\begin{thm}
Let $ \phi:G\to G' $ be an isomorphism. Then, $ \phi^{-1} $ is an isomorphism.
\end{thm}
\begin{proof}
Let $ \odot $ denote the law of composition for group $ G $ and $ \oslash $ denote the law for $ G' $, let $ f=\phi^{-1} $, and let $ x,y\in G' $.  $ f $ is clearly well-defined, and we see
\begin{equation*}
    \phi(f(x)\odot f(y))=\phi(f(x))\oslash\phi(f(y))=x\oslash y=\phi(f(x\oslash y)).
\end{equation*}
Since $ \phi $ is injective, this implies $ f(x)\odot f(y)=f(x\oslash y) $, so $ f $ is a homomorphism. Injectivity and surjectivity can be easily verified. Hence, $ f $ is an isomorphism.
\end{proof}

\begin{thm}[Fundamental theorem on homomorphisms]
Let $ \phi:G\to G' $ be a homomorphism. Then, the mapping
\begin{equation*}
    \begin{array}{rccc}
        \psi: & G/\ker(\phi) & \to & \im(\phi) \\
        & g\ker(\phi) & \mapsto & \phi(g)
    \end{array}
\end{equation*}
is an isomorphism.
\end{thm}
\begin{proof}
We have four criteria for $ \psi $ to be an isomorphism:
\begin{enumerate}
    \item Let $ g,h $ be such that $ g\ker(\phi)=h\ker(\phi) $. Then, $ h^{-1}g\in\ker(\phi) $, so
    \begin{align*}
        \phi(h^{-1}g) &= e_{G'} \\
        (\phi(h))^{-1}\phi(g) &= e_{G'} \\
        \phi(g) &= \phi(h).
    \end{align*}
    Thus, $ \psi $ is well-defined.

    \item Let $ g\ker(\phi),h\ker(\phi)\in G/\ker(\phi) $. Then,
    \begin{align*}
        \psi(g\ker(\phi)\,h\ker(\phi)) &= \psi((gh)\ker(\phi))=\phi(gh)=\phi(g)\,\phi(h) \\
        &= \psi(g\ker(\phi))\,\psi(h\ker(\phi)),
    \end{align*}
    so $ \psi $ is a homomorphism.

    \item Let $ g\ker(\phi)\in\ker(\psi) $. Then, $ \psi(g\ker(\phi))=e_{G'} $, so $ g\in\ker(\phi) $, which implies $ g\ker(\phi)=\ker(\phi) $. Thus, by Theorem \ref{thm:monomorphism}, $ \psi $ is injective.

    \item $ \psi $ is surjective by construction since it maps to $ \im(\phi) $.
\end{enumerate}
Hence, $ \psi $ is an isomorphism.
\end{proof}

This theorem is also known as the first isomorphism theorem.

\section{Permutation groups}

\begin{prop}\label{prop:permutations}
Let $ X $ be a set, and let $ \mathcal{S}(X) $ be the set of all bijections from $ X $ to $ X $. Then, $ (\mathcal{S}(X),\circ) $, where $ \circ $ is composition of mappings, is a group.
\end{prop}
\begin{proof}
\todo{Do this proof!}
\end{proof}

\begin{defn}
Take $ \mathcal{S}(X) $ as defined in Proposition \ref{prop:permutations} for some set $ X $. A subgroup of $ \mathcal{S}(X) $ is called a \defnem{permutation group}. Any mapping in such a group is called a \defnem{permutation}.
\end{defn}

The neutral element of a permutation group is naturally the identity mapping, which we will denote $ \id $.

\begin{defn}\label{defn:symmetric_group}
Let $ A=\{1,2,\ldots,n\} $ for some $ n\in\mathbb{N} $. Then, $ \mathcal{S}_n=\mathcal{S}(A) $ is called the \defnem{symmetric group} on $ n $ elements.
\end{defn}

More generally, $ \mathcal{S}_n $ can be used to describe the group of permutations of any finite set. Since any finite set is isomorphic to a subset of $ \mathbb{N} $, we can apply this definition by assigning a label in $ A $ to each element. The results we will show for $ \mathcal{S}_n $ therefore apply with this generalization as well.

Note that for any $ n\in\mathbb{N} $, we have $ \lvert\mathcal{S}_n\rvert=n! $. This may be familiar if you recall the notion of a permutation of a set as a rearrangement of its elements. Consider the following permutation $ \sigma\in\mathcal{S}_5 $:
\begin{align*}
    1 &\mapsto 3 \\
    2 &\mapsto 2 \\
    3 &\mapsto 5 \\
    4 &\mapsto 4 \\
    5 &\mapsto 1.
\end{align*}
We will represent it with the notation
\begin{equation*}
    \sigma=
    \begin{pmatrix}
        1 & 2 & 3 & 4 & 5 \\
        3 & 2 & 5 & 4 & 1
    \end{pmatrix}.
\end{equation*}

\begin{defn}
Let $ \sigma=\mathcal{S}_n $. The set
\begin{equation*}
    \supp(\sigma)=\{i\in\{1,2,\ldots,n\}\mid\sigma(i)\neq i\}
\end{equation*}
is called the \defnem{support} of $ \sigma $.
\end{defn}

\begin{prop}
Let $ \sigma,\tau\in\mathcal{S}_n $. If $ \supp(\sigma)\cap\supp(\tau)=\varnothing $, then $ \sigma\circ\tau=\tau\circ\sigma $.
\end{prop}
\begin{proof}
Let $ i\in\{1,2,\ldots,n\} $. We have three cases:
\begin{enumerate}
    \item Suppose $ i\notin\supp(\sigma)\cup\supp(\tau) $. Then, $ \sigma(i)=\tau(i)=i $, so
    \begin{align*}
        (\sigma\circ\tau)(i) &= \sigma(\tau(i))=\sigma(i)=i=\tau(i)=\tau(\sigma(i))=(\tau\circ\sigma)(i).
    \end{align*}

    \item Suppose $ i\in\supp(\sigma) $. Then, $ i\notin\supp(\tau) $, so
    \begin{equation*}
        (\sigma\circ\tau)(i)=\sigma(\tau(i))=\sigma(i),
    \end{equation*}
    and since $ i\in\supp(\sigma) $, we have $ \sigma(i)\in\supp(\sigma) $, so $ \sigma(i)\notin\supp(\tau) $. Thus,
    \begin{equation*}
        (\tau\circ\sigma)(i)=\tau(\sigma(i))=\sigma(i)=(\sigma\circ\tau)(i).
    \end{equation*}

    \item If $ i\in\supp(\tau) $, the proof can be done in the same way as in the above case.
\end{enumerate}
Hence, $ \sigma\circ\tau=\tau\circ\sigma $.
\end{proof}

\begin{thm}[Cayley's theorem]
Every group is isomorphic to a permutation group.
\end{thm}
\begin{proof}
    \todo{Do this proof!}
\end{proof}

\subsection*{Cycles}

\begin{defn}
An element $ \sigma\in\mathcal{S}_n $ is called a \defnem{cycle} if there exists some $ x\in\{1,2,\ldots,n\} $ such that $ \supp(\sigma)=\{\sigma^i(x)\mid i\in\mathbb{N}\} $. Let $ l=\lvert\supp(\sigma)\rvert $. We denote the cycle
\begin{equation*}
    \left(x,\sigma(x),\ldots,\sigma^{l-1}(x)\right)
\end{equation*}
where $ l $ is called its \defnem{length}. A cycle of length 2 is called a \defnem{transposition}.
\end{defn}

\begin{prop}
Let $ \sigma $ be a cycle of length $ l $. Then, $ \ord(\sigma)=l $.
\end{prop}

This follows by construction.

\begin{prop}\label{prop:orbits}
Let $ \sigma\in\mathcal{S}_n $, and let $ A=\{1,2,\ldots,n\} $. Then, the relation $ \sim $ on $ A $ defined such that for all $ a,b\in A $,
\begin{equation*}
    a\sim b \iff \text{there exists some }k\in\mathbb{Z}\text{ such that }b=\sigma^k(a)
\end{equation*}
is an equivalence relation.
\end{prop}
\begin{proof}
We have three criteria for an equivalence relation:
\begin{enumerate}
    \item Since $ a=\sigma^0(a) $, we have $ a\sim a $ (reflexive).
    \item Suppose $ a\sim b $. Then, $ b=\sigma^k(a) $ for some $ k\in\mathbb{Z} $, so $ a=\sigma^{-k}(b) $. Thus, $ b\sim a $ (symmetric).
    \item Let $ c\in A $. Suppose $ a\sim b $ and $ b\sim c $. Then, $ b=\sigma^k(a) $ and $ c=\sigma^m(b) $ for some $ k,m\in\mathbb{Z} $, so $ c=\sigma^m(\sigma^k(a))=\sigma^{m+k}(a) $. Thus, $ a\sim c $ (transitive).\qedhere
\end{enumerate}
\end{proof}

\begin{cor}[Alternative definition of a cycle]
Take $ \sim $ as defined in Proposition \ref{prop:orbits} for some $ \sigma\in\mathcal{S}_n $. Then, $ \sigma $ is a cycle if and only if $ \sim $ has at most one equivalence class containing more than one element.
\end{cor}

\begin{thm}\label{thm:cycle_decomp}
Let $ \sigma\in\mathcal{S}_n $. Then, there exist some unique cycles $ \tau_1,\tau_2,\ldots,\tau_k $ with disjoint supports such that $ \sigma=\tau_1\circ\tau_2\circ\cdots\circ\tau_k $. In other words, every permutation of a finite set can be decomposed as the product of unique cycles with disjoint supports.
\end{thm}
\begin{proof}
Let $ A_1,A_2,\ldots,A_k $ be the equivalence classes of $ \sim $, and let $ \tau_1,\tau_2,\ldots,\tau_k $ be the cycles defined such that 
\begin{equation*}
    \tau_i(x)=
    \left\{\begin{array}{ll}
        \sigma(x), & x\in A_i \\
        x, & \text{otherwise.}
    \end{array}\right.
\end{equation*}
We see $ \sigma=\tau_1\circ\tau_2\circ\cdots\circ\tau_k $, and since $ A_1,A_2,\ldots,A_k $ are necessarily disjoint, $ \tau_1,\tau_2,\ldots,\tau_k $ have disjoint supports.
\end{proof}

\begin{defn}
Let $ \sigma\in\mathcal{S}_n $ with decomposition $ \sigma=\tau_1\circ\tau_2\circ\cdots\circ\tau_k $ as given by Theorem \ref{thm:cycle_decomp}. Let $ l_1,l_2,\ldots,l_k $ denote the lengths of $ \tau_1,\tau_2,\ldots,\tau_k $, respectively, where $ l_1\geq l_2\geq\cdots\geq l_k $. The sequence $ (l_1,l_2,\ldots,l_k) $ is called the \defnem{type} of $ \sigma $.
\end{defn}

\begin{prop}
Let $ \sigma\in\mathcal{S}_n $ with type $ (l_1,l_2,\ldots,l_k) $. Then,
\begin{equation*}
    \ord(\sigma)=\lcm\{l_1,l_2,\ldots,l_k\}.
\end{equation*}
\end{prop}
\begin{proof}
We can decompose $ \sigma $ into cycles as $ \sigma=\tau_1\circ\tau_2\circ\cdots\circ\tau_k $ where $ \tau_1,\tau_2,\ldots,\tau_k $ have length $ l_1,l_2,\ldots,l_k $, respectively. Since the $ \tau_i $s have disjoint supports, they commute, so for every $ m\in\mathbb{N} $, we have
\begin{equation*}
    \sigma^m=\tau_1^m\circ\tau_2^m\circ\cdots\circ\tau_k^m.
\end{equation*}
Since $ \ord(\tau_i)=l_i $ for $ 1\leq i\leq k $, we see that if $ \sigma^m=\id $, then $ m $ is a multiple of each of the $ l_i $s. Hence, by definition, $ \ord(\sigma) $ is the lowest such $ m $.
\end{proof}

\subsection*{Transpositions and alternating groups}

\begin{cor}[to Theorem \ref{thm:cycle_decomp}]
Every permutation in $ \mathcal{S}_n $ can be decomposed as the product of transpositions.
\end{cor}

\begin{prop}
Let $ \sigma\in\mathcal{S}_n $. Either all transposition decompositions of $ \sigma $ are the product of an even number of transpositions, or all of them are the product of an odd number of transpositions.
\end{prop}
\begin{proof}
Consider the group $ \mathcal{S}_{I,n} $ of permutations of the rows of the $ n\times n $ identity matrix $ I_n $. As remarked following Definition \ref{defn:symmetric_group}, $ \mathcal{S}_{I,n}\simeq\mathcal{S}_n $. We know $ \det(I_n)=1 $, and transposing any two rows of a square matrix changes the sign of its determinant.

Let $ \sigma\in\mathcal{S}_{I,n} $, and let $ A=\sigma(I_n) $. Suppose $ \sigma $ can be decomposed as an even number of transpositions. Then, $ \det(A)=1 $. Now suppose $ \sigma $ can also be decomposed as an odd number of transpositions. Then, $ \det(A)=-1 $, a contradiction. Hence, no $ \sigma\in\mathcal{S}_{I,n} $ can be decomposed into the product of both an even number and an odd number of transpositions.
\end{proof}

\begin{defn}
Let $ \sigma\in\mathcal{S}_n $, and let $ k $ be the number of transpositions in some transposition decomposition of $ \sigma $. The number $ \epsilon(\sigma)=(-1)^k $ is called the \defnem{signature} of $ \sigma $. The permutation $ \sigma $ is called \defnem{even} if $ k $ is even or \defnem{odd} if $ k $ is odd.
\end{defn}

\begin{prop}\label{prop:alternating_group}
Let $ \mathcal{A}_n=\{\sigma\in\mathcal{S}_n\mid\epsilon(\sigma)=1\} $. Then, $ \mathcal{A}_n $ is a normal subgroup of $ \mathcal{S}_n $.
\end{prop}
\begin{proof}
Let $ \alpha\in\mathcal{A}_n $ and $ \sigma\in\mathcal{S}_n $. For some $ k,m\in\mathbb{N} $, $ \alpha $ can be decomposed as the product of some number $ 2k $ of transpositions and $ \sigma $ can be decomposed as the product of some number $ m $ of transpositions, so there exists a decomposition of $ \sigma\circ\alpha\circ\sigma^{-1} $ into some number $ m+2k+m=2(m+k) $ of transpositions. Hence, $ \sigma\circ\alpha\circ\sigma^{-1}\in\mathcal{A}_n $, so by Theorem \ref{prop:normal_subgroup}, $ \mathcal{A}_n $ is a normal subgroup of $ \mathcal{S}_n $.
\end{proof}

We can alternatively show that the mapping
\begin{equation*}
    \begin{array}{rccc}
        \epsilon: & (\mathcal{S}_n,\circ) & \to & (\{-1,1\},\cdot) \\
        & \sigma & \mapsto & \epsilon(\sigma)
    \end{array}
\end{equation*}
is a group homomorphism and that $ \mathcal{A}_n=\ker(\epsilon) $. By Theorem \ref{thm:ker_normal_subgroup}, this implies $ \mathcal{A}_n $ is a normal subgroup of $ \mathcal{S}_n $.

\begin{defn}
$ \mathcal{A}_n $ as defined in Proposition \ref{prop:alternating_group} is called the \defnem{alternating group} on $ n $ elements.
\end{defn}

\subsection*{The dihedral group}

\section{Finitely generated abelian groups}

\section{Group action on a set}